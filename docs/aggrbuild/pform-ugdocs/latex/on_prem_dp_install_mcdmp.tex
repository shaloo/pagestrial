%% Generated by Sphinx.
\def\sphinxdocclass{report}
\documentclass[letterpaper,10pt,english]{sphinxmanual}
\ifdefined\pdfpxdimen
   \let\sphinxpxdimen\pdfpxdimen\else\newdimen\sphinxpxdimen
\fi \sphinxpxdimen=.75bp\relax

\usepackage[utf8x]{inputenc}

\usepackage{cmap}
\usepackage[T1]{fontenc}
\usepackage{amsmath,amssymb,amstext}
\usepackage{babel}
\usepackage{times}
\usepackage[Bjarne]{fncychap}
\usepackage[dontkeepoldnames]{sphinx}

\usepackage{geometry}

% Include hyperref last.
\usepackage{hyperref}
% Fix anchor placement for figures with captions.
\usepackage{hypcap}% it must be loaded after hyperref.
% Set up styles of URL: it should be placed after hyperref.
\urlstyle{same}

\addto\captionsenglish{\renewcommand{\figurename}{Fig.}}
\addto\captionsenglish{\renewcommand{\tablename}{Table}}
\addto\captionsenglish{\renewcommand{\literalblockname}{Listing}}

\addto\captionsenglish{\renewcommand{\literalblockcontinuedname}{continued from previous page}}
\addto\captionsenglish{\renewcommand{\literalblockcontinuesname}{continues on next page}}

\addto\extrasenglish{\def\pageautorefname{page}}



\usepackage{enumitem}

\title{Loom On-Premises Data Plane Deployment Guide}
\date{Apr 07, 2018}
\release{0.7.0}
\author{Veritas Technologies LLC}
\newcommand{\sphinxlogo}{\sphinxincludegraphics{loom.png}\par}
\renewcommand{\releasename}{Release}
\makeindex

\begin{document}

\maketitle
\sphinxtableofcontents
\phantomsection\label{\detokenize{on_prem_dp_install_mcdmp::doc}}


\begin{sphinxShadowBox}
\begin{itemize}
\item {} 
\phantomsection\label{\detokenize{on_prem_dp_install_mcdmp:id1}}{\hyperref[\detokenize{on_prem_dp_install_mcdmp:about-data-engine}]{\sphinxcrossref{About Data Engine}}}

\item {} 
\phantomsection\label{\detokenize{on_prem_dp_install_mcdmp:id2}}{\hyperref[\detokenize{on_prem_dp_install_mcdmp:system-requirements}]{\sphinxcrossref{System requirements}}}

\item {} 
\phantomsection\label{\detokenize{on_prem_dp_install_mcdmp:id3}}{\hyperref[\detokenize{on_prem_dp_install_mcdmp:prerequisites-for-deploying-an-on-premises-data-engine}]{\sphinxcrossref{Prerequisites for deploying an on-premises Data Engine}}}

\item {} 
\phantomsection\label{\detokenize{on_prem_dp_install_mcdmp:id4}}{\hyperref[\detokenize{on_prem_dp_install_mcdmp:high-level-workflow-of-deploying-on-premises-data-engine}]{\sphinxcrossref{High-level workflow of deploying on-premises Data Engine}}}

\item {} 
\phantomsection\label{\detokenize{on_prem_dp_install_mcdmp:id5}}{\hyperref[\detokenize{on_prem_dp_install_mcdmp:downloading-the-on-premises-installer-kit}]{\sphinxcrossref{Downloading the on-premises installer kit}}}

\item {} 
\phantomsection\label{\detokenize{on_prem_dp_install_mcdmp:id6}}{\hyperref[\detokenize{on_prem_dp_install_mcdmp:setting-up-a-virtual-machine-environment}]{\sphinxcrossref{Setting up a virtual machine environment}}}

\item {} 
\phantomsection\label{\detokenize{on_prem_dp_install_mcdmp:id7}}{\hyperref[\detokenize{on_prem_dp_install_mcdmp:deploying-data-engine-using-the-wizard}]{\sphinxcrossref{Deploying Data Engine using the wizard}}}

\item {} 
\phantomsection\label{\detokenize{on_prem_dp_install_mcdmp:id8}}{\hyperref[\detokenize{on_prem_dp_install_mcdmp:installing-the-connector-agent}]{\sphinxcrossref{Installing the Connector Agent}}}

\item {} 
\phantomsection\label{\detokenize{on_prem_dp_install_mcdmp:id9}}{\hyperref[\detokenize{on_prem_dp_install_mcdmp:troubleshooting-on-premises-data-engine-deployment}]{\sphinxcrossref{Troubleshooting on-premises Data Engine deployment}}}

\end{itemize}
\end{sphinxShadowBox}


\chapter{About Data Engine}
\label{\detokenize{on_prem_dp_install_mcdmp:abt-dataengine}}\label{\detokenize{on_prem_dp_install_mcdmp:loom-on-premise-data-plane-deployment}}\label{\detokenize{on_prem_dp_install_mcdmp:about-data-engine}}\label{\detokenize{on_prem_dp_install_mcdmp:ing-com-on-prem-dp-install}}
The Loom platform constitutes of one Control Plane and Data Engine(s) required for intelligently monitoring your data. Data Engine comprises of several
micro-services which are critical for capabilities such as logging, monitoring, and data processing. Data Engine is tightly coupled with Control Plane in terms of
monitoring and managing enterprise information distributed across cloud and on-premises data sources.

Data Engine is set up to use the Veritas Data Connector framework that ties down various enterprise data sources (CIFS, NFS, Box, and others) to the Loom
deployment. In conjunction with the other Loom components, Data Engine inconspicuously works towards offering visualization and governance of enterprise
data assets.

The following image shows the key components in the on-premises Data Engine and Control Plane environment.

\begin{figure}[htbp]
\centering

\noindent\sphinxincludegraphics[scale=1.0]{{abt_on-prem_dp_components}.png}
\end{figure}

The on-premises Data Engine architecture includes the following components:


\begin{savenotes}\sphinxattablestart
\centering
\begin{tabulary}{\linewidth}[t]{|T|T|}
\hline

\sphinxstylestrong{Components}
&
\sphinxstylestrong{Description}
\\
\hline
Control Plane
&
Hosts micro-services that are
responsible for managing and scheduling
jobs on Data Engine. Stores the data
collected from the content sources.
\\
\hline
Data Engine
&
Interacts with the content sources
such as Microsoft SharePoint, file
servers, to initiate scan and discovery
collectmetadata, and transmit it to
Control Plane.
\\
\hline
Connector Agent
&
Discovers data across various content
residing in cloud and on-premises
content sources, and
Veritas-integrated connectors.
\\
\hline
\end{tabulary}
\par
\sphinxattableend\end{savenotes}


\chapter{System requirements}
\label{\detokenize{on_prem_dp_install_mcdmp:sys-req}}\label{\detokenize{on_prem_dp_install_mcdmp:system-requirements}}
This section lists the system requirements that must be met to deploy the on-premises Data Engine.
\begin{itemize}
\item {} 
Data Engine must be deployed on virtual machines hosted on VMWare ESX Server. The minimum ESX version must be 5.5.

\item {} 
The virtual machines (master and worker nodes) on which Data Engine is deployed must have Ubuntu 16.0.4 or Red Hat Enterprise Linux version 7.4 or
higher operating system installed.

\item {} 
The operating system for Connector Agent must be Windows 2012, 2012 R2, or Windows 2016. The operating system must be 64 bit.

\item {} 
The data at rest is stored on the file server. The responsibility for encrypting the data on file server lies with your organization.

\item {} 
Each server requires one network interface card (NIC).

\item {} 
The communication between Data Engine and Control Plane happens through ports 443 and 80. Thus, these ports must be open and accessible from Data
Engine.

\end{itemize}


\begin{savenotes}\sphinxattablestart
\centering
\begin{tabular}[t]{|*{2}{\X{1}{2}|}}
\hline
\begin{description}
\item[{Virtual machines for deploying}] \leavevmode
Data Engine

\end{description}
&\begin{itemize}
\item {} 
1 master node and 3 worker nodes

\item {} 
8 cores and 16 GB RAM on each node

\item {} 
Disk space for shared volume -
100GB

\item {} 
Operating system - Ubuntu or Red Hat
Enterprise Linux

\end{itemize}
\\
\hline
Virtual machine for Connector Agent
&\begin{itemize}
\item {} 
4 cores

\item {} 
8 GB RAM

\item {} 
Disk space for shared volume-100 GB

\item {} 
Operating system - Windows

\end{itemize}
\\
\hline
\end{tabular}
\par
\sphinxattableend\end{savenotes}


\chapter{Prerequisites for deploying an on-premises Data Engine}
\label{\detokenize{on_prem_dp_install_mcdmp:prerequisites-for-deploying-an-on-premises-data-engine}}
The on-premises Data Engine is deployed on the VMware ESX Server. For successful deployment of the on-premises Data Engine, ensure that the following
prerequisites are satisfied:
\begin{itemize}
\item {} 
You need privileges on vCenter Server to deploy OVF template. Consult the VMWare vSphere documentation for more information about the privileges.

\item {} 
At least four nodes should be configured on the VMware ESX Server. These nodes are utilized to deploy various Data Engine services. The recommended
cluster configuration comprises of one master node and three worker nodes. An additional node must be allocated to deploy the Open Virtual Appliance
(OVA) file.

\item {} 
Each node must have the following configuration:
\begin{itemize}
\item {} 
The starter virtual machine should be able to access the virtual machines (master and worker nodes). A starter virtual machine is an independent
machine which hosts the Airflow pod that is used to deploy Data Engine on multiple virtual machines.

\item {} 
The swap functionality should be disabled on all the nodes. If the swap functionality is enabled, then the kubernetes cluster fails to restart.

To disable the swap functionality, run the swapoff -a command.

To disable the swap functionality that applies even after a restart, remove the swap entry from the /etc/fstab file.

\item {} 
All nodes must synchronize their system time to a NTP server.

\end{itemize}

\item {} 
On all nodes, root access should be enabled to permit the root user to log on to the nodes using SSH.

To enable root access, do the following on each node:
\begin{enumerate}
\item {} 
In the /etc/ssh/sshd\_config file, set PermitRootLogin to yes.

\item {} 
Restart the SSH service using the “/etc/init.d/ssh restart” command.

\end{enumerate}

\item {} 
The file server packages must be installed on all nodes using the following commands:

On RHEL, run the following command:

yum install nfs-utils

On Ubuntu, do the following:
\begin{enumerate}
\item {} 
Run the apt-get install nfs-common command on all nodes.

\item {} 
Run the apt-get install nfs-kernel-server command on the master node.

\end{enumerate}

\item {} 
All the nodes in the Kubernetes cluster must mount the same file server directory that has the same path.

To ensure that the file server mount path persists across restarts, add the following in the /etc/fstab file:

\textless{}NFS Server IP\textgreater{}/export/data /mnt/disk nfs rw 0 0

Restart the computer to confirm if the file server is successfully mounted.

\item {} 
The IP routing table of the virtual machines must not have address prefixes that begin with 10.x.x.

To view the route table contents, run route -n command.

To delete an entry in the routing table, run:
route del \textendash{}net \textless{}IP address\textgreater{} netmask \textless{}network mask value\textgreater{} dev \textless{}network interface name\textgreater{}

For example, route del \textendash{}net 10.0.0.0 netmask 255.0.0.0 dev ens160

\item {} 
All nodes in the cluster must have the same root password. This is to ensure that the nodes are accessible from the starter virtual machine. Once the on-premises Data Engine is deployed, you can reset the root password.

\item {} 
For successful communication between the nodes in the Kubernetes cluster, the firewall settings should be disabled on all nodes.
To disable firewall daemon on all nodes, run the following command as a root user:
systemctl disable firewalld

\end{itemize}


\chapter{High-level workflow of deploying on-premises Data Engine}
\label{\detokenize{on_prem_dp_install_mcdmp:high-level-workflow-of-deploying-on-premises-data-engine}}
The on-premises Data Engine is deployed in a distributed environment across multiple servers to improve operational efficiency and facilitate scalability. Data
Engine can be deployed on cloud and on-premises data infrastructure. You must configure the on-premises Data Engine depending upon the type of content sources
that you intend to monitor such as on-premises Data Engine for Microsoft SharePoint and file server, and SaaS engine for Microsoft SharePoint Online and Box, and
other cloud-based content sources. Users with Tenant Administrator’s role are responsible for performing the deployment tasks.

\begin{sphinxadmonition}{note}{Note:}
Before deploying the engine, ensure that the SaaS Control Plane is deployed.
\end{sphinxadmonition}

The on-premises Data Engine deployment takes place on virtual machines hosted on VMware ESX Server. The deployment procedure is split into four high-level
steps:
\begin{enumerate}
\item {} 
Download the on-premises Data Engine installer kit from the Loom UI. See {\hyperref[\detokenize{on_prem_dp_install_mcdmp:download-install-kit}]{\sphinxcrossref{\DUrole{std,std-ref}{Downloading the on-premises installer kit}}}}.

\end{enumerate}

\begin{sphinxadmonition}{note}{Note:}
For Alpha 2 release, Veritas will provide the Data Engine installer kit. You are not required to download the installer kit.
\end{sphinxadmonition}
\begin{enumerate}
\setcounter{enumi}{1}
\item {} 
Set up your VMWare ESX environment to deploy the on-premises Data Engine. See {\hyperref[\detokenize{on_prem_dp_install_mcdmp:set-up-vm-env}]{\sphinxcrossref{\DUrole{std,std-ref}{Setting up a virtual machine environment}}}}.

\item {} 
Run the Data Engine deployment wizard to complete the deployment. See {\hyperref[\detokenize{on_prem_dp_install_mcdmp:depl-dp-wizard}]{\sphinxcrossref{\DUrole{std,std-ref}{Deploying Data Engine using the wizard}}}}.

\item {} 
Install Connector Agent on a Windows virtual machine. This is an optional step. It must be performed if you intend to monitor Microsoft SharePoint and file
servers, such as NetApp, Windows File Servers, EMC VNX, EMC Isilon.  See {\hyperref[\detokenize{on_prem_dp_install_mcdmp:install-dp-agent}]{\sphinxcrossref{\DUrole{std,std-ref}{Installing the Connector Agent}}}}.

\end{enumerate}

The following image provides a high-level workflow of deploying on-premises Data Engine.

\begin{figure}[htbp]
\centering

\noindent\sphinxincludegraphics[scale=1.0]{{on-prem_deployment}.png}
\end{figure}


\chapter{Downloading the on-premises installer kit}
\label{\detokenize{on_prem_dp_install_mcdmp:downloading-the-on-premises-installer-kit}}\label{\detokenize{on_prem_dp_install_mcdmp:download-install-kit}}
For Alpha 2, Veritas will provide the Data Engine installer kit. This implies that you are not required to download the installer kit.

From the Loom interface, you must download the installation kit for the on-premises Data Engine.

\sphinxstylestrong{To download the installer kit}
\begin{enumerate}
\item {} 
Log on to the Loom UI with the Tenant Administrator credentials.

\item {} 
In the left navigation pane of the Loom UI, click \sphinxstylestrong{Data Engines}.
The list of configured engines is displayed.

\item {} 
Click \sphinxstylestrong{Add}.

\item {} 
On the \sphinxstylestrong{Data Engine Installation} wizard window, click \sphinxstylestrong{VMWare On-Premises}, and click \sphinxstylestrong{Next}.

\item {} 
On the \sphinxstylestrong{Register Data Engine} panel, enter a unique name for your engine. A valid Data Engine name should be in lower case and contain at least 5, but not more than 12 characters.

\item {} 
Enter the location details for your on-premises Data Engine cluster.

\item {} 
Click \sphinxstylestrong{Register Data Engine} to register Data Engine with Control Plane. The Registration (starter) file that contains the registration details is downloaded.

\item {} 
Click \sphinxstylestrong{VMware Installer Kit}. The installer kit consists of an OVA file, and Windows installer file. The .OVA file is used to create a Stater virtual machine
in your environment. You are directed to the Data Engine list page while the download is in progress. If you have already installed a Data Engine before,
click \sphinxstylestrong{I have already installed a Data Engine deployment app} and proceed to the Setting up a virtual machine environment for the next steps.

\end{enumerate}

If you encounter errors when downloading the Registration file or the VMware installer, you can download them from the Data Engine list page on the Loom UI.

Once the VMware installer kit has finished downloading, the \sphinxstylestrong{Data Engine} list page displays the \sphinxstylestrong{Deployment Status} for your Data Engine as \sphinxstylestrong{In Progress}.


\chapter{Setting up a virtual machine environment}
\label{\detokenize{on_prem_dp_install_mcdmp:setting-up-a-virtual-machine-environment}}\label{\detokenize{on_prem_dp_install_mcdmp:set-up-vm-env}}
A virtual machine cluster can be configured through a client, such as VMware vSphere Client.

This section describes the steps for creating a virtual machine cluster using VMware vSphere Client. If you plan to use any other client service, refer to the
corresponding documentation.

Review the system requirements for the VMWare ESX Server before creating the virtual machines. See {\hyperref[\detokenize{on_prem_dp_install_mcdmp:sys-req}]{\sphinxcrossref{\DUrole{std,std-ref}{System requirements}}}}

\sphinxstylestrong{To set up VMware environment}
\begin{enumerate}
\item {} 
Log on to the VMware vSphere Client, click \sphinxstylestrong{File} \textgreater{} \sphinxstylestrong{Deploy OVF Template}.

\item {} 
Browse to the downloaded OVA file, enter the required network credentials, and start the starter image.

\item {} 
Click \sphinxstylestrong{Select your operating system} drop-down to choose the operating system that you want to install on the virtual machines.

\item {} 
Right-click the newly added machine, click \sphinxstylestrong{Power On}, and assign an IP address to the virtual machine.
In this step, the contents of the OVA file get extracted. Usually, it takes several minutes to complete the extraction. The progress bar on the wizard confirms when the extraction is complete.

\end{enumerate}


\chapter{Deploying Data Engine using the wizard}
\label{\detokenize{on_prem_dp_install_mcdmp:depl-dp-wizard}}\label{\detokenize{on_prem_dp_install_mcdmp:deploying-data-engine-using-the-wizard}}
After the OVA file is extracted, you must complete the deployment procedure by following the deployment wizard.

\sphinxstylestrong{To run the deployment wizard}
\begin{enumerate}
\item {} 
On the web browser, enter the IP address of the newly-created starter virtual machine. This opens the Data Engine configuration UI.

\item {} 
On the \sphinxstylestrong{Data Engine Configuration} wizard \textgreater{} \sphinxstylestrong{Configure Agent} screen, browse to the Registration file that you downloaded from the Loom UI.

\item {} 
Enter the following details to register the Data Engine with the platform:

\end{enumerate}
\begin{itemize}
\item {} 
Master node IP

\item {} 
Worker node IP addresses - Enter multiple IP addresses separated by commas.

\item {} 
Storage mount path

\item {} 
VM root password. This password is used to connect to the virtual machines that you have specified. Note that you can reset the password after logging on to the virtual
machines.

\end{itemize}
\begin{enumerate}
\setcounter{enumi}{3}
\item {} 
Click \sphinxstylestrong{Next} to view the \sphinxstylestrong{Configure Network} screen. You can configure the proxy server that you want to use to connect to the Internet.

\item {} 
If your organization requires Internet connections to pass through a proxy server, select one of the following:

\end{enumerate}
\begin{quote}
\begin{itemize}
\item {} 
Direct: Use this option if you intend to access the Internet without using a proxy server.

\item {} 
Proxy with NTLM authentication

\item {} 
Proxy with default authentication

\item {} 
Proxy without authentication

\end{itemize}

Depending upon the proxy server that you select, enter the following details:
\end{quote}
\begin{itemize}
\item {} 
The domain server name of the IP address of the proxy server host.

\item {} 
The details of the proxy port.

\item {} 
In case of connection via proxy with NTLM or Default authentication, enter a valid user account and password. The password should be set to not expire. The password is encrypted when stored. Click \sphinxstylestrong{Test Connection} to ensure that you can connect to the proxy server with the credentials you
have provided. Click \sphinxstylestrong{Next}.

\end{itemize}
\begin{enumerate}
\setcounter{enumi}{5}
\item {} 
On the \sphinxstylestrong{Confirm \& Deploy} screen, review the details of the cluster and click \sphinxstylestrong{Deploy}.

\end{enumerate}

The \sphinxstylestrong{Monitor Deployment} screen shows the status of the deployment progress. After the machines in the cluster have been deployed and app services started,
you will be notified that Data Engine is successfully deployed. You can return to the Loom UI to review the status of the deployment.


\chapter{Installing the Connector Agent}
\label{\detokenize{on_prem_dp_install_mcdmp:installing-the-connector-agent}}\label{\detokenize{on_prem_dp_install_mcdmp:install-dp-agent}}
\begin{sphinxadmonition}{note}{Note:}
You must install the Connector Agent only if you want to monitor Microsoft SharePoint online and native file system content sources.
\end{sphinxadmonition}

The connector Agent installer is downloaded as a part of the deployment installer kit from the Loom UI. You must provision a Windows virtual machine to complete
the installation.

See {\hyperref[\detokenize{on_prem_dp_install_mcdmp:sys-req}]{\sphinxcrossref{\DUrole{std,std-ref}{System requirements}}}}

\sphinxstylestrong{To install the Connector Agent on a Windows virtual machine}
\begin{enumerate}
\item {} 
Log on to the machine intended to serve as the Agent server with the administrator credentials.

\item {} 
Extract the zip package contents and double-click \textless{}Veritas\_DPAgent\_windows\_\textless{}build\textgreater{}\_x64.exe to launch the installer. If you are not logged on to the computer as Administrator, you will be prompted to enter the Administrator credentials.

\item {} 
The \sphinxstylestrong{Welcome to the Setup Wizard} window appears. Click \sphinxstylestrong{Next}.

\item {} 
In the License Agreement panel, select \sphinxstylestrong{I accept the agreement} and click \sphinxstylestrong{Next}.

\item {} 
On the \sphinxstylestrong{Agent Configuration} details panel, enter the following details:

\end{enumerate}
\begin{itemize}
\item {} 
Ingress IP  - Enter the IP addresses of the master and worker nodes with which Connector Agent must communicate with.

\item {} 
Browse to the location where you want to store the product data before it is sent to the Control Plane. By default, the destination directory is C:VeritasDPpAgentdata. You can configure it in a separate directory.
The data directory should reside on a volume that has sufficient space and high-performance disk.

\end{itemize}
\begin{enumerate}
\setcounter{enumi}{5}
\item {} 
Click \sphinxstylestrong{Next} to begin the installation.

\item {} 
Click \sphinxstylestrong{Finish}.

\end{enumerate}


\chapter{Troubleshooting on-premises Data Engine deployment}
\label{\detokenize{on_prem_dp_install_mcdmp:troubleshooting-on-premises-data-engine-deployment}}
\sphinxstylestrong{Table: Troubleshooting steps}


\begin{savenotes}\sphinxattablestart
\centering
\begin{tabulary}{\linewidth}[t]{|T|T|}
\hline
\sphinxstylethead{\sphinxstyletheadfamily 
Error
\unskip}\relax &\sphinxstylethead{\sphinxstyletheadfamily 
Solution
\unskip}\relax \\
\hline
An error occurs when
downloading the
VMWareDeploymentKit.zip
&
Click Try Again on the error pop-up,
or try to download the zip file from
the Data Engine list page on the Loom
UI.
\\
\hline
An error occurs when
deploying  the cluster
&
Click \sphinxstylestrong{Redeploy} or download the
Airflow log from the link on the
wizard and sent it to Customer Support
\\
\hline
When the OVA file is
uploaded to create a
virtual environment,
the contents of the
file are extracted.
If any of the
components becomes
corrupt when the
extraction is
in-progress, an error
message is reported on
the Loom UI.
&
Restart the OVA deployment process.
\\
\hline
\end{tabulary}
\par
\sphinxattableend\end{savenotes}



\renewcommand{\indexname}{Index}
\printindex
\end{document}