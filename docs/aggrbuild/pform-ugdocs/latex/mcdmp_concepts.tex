%% Generated by Sphinx.
\def\sphinxdocclass{report}
\documentclass[letterpaper,10pt,english]{sphinxmanual}
\ifdefined\pdfpxdimen
   \let\sphinxpxdimen\pdfpxdimen\else\newdimen\sphinxpxdimen
\fi \sphinxpxdimen=.75bp\relax

\usepackage[utf8x]{inputenc}

\usepackage{cmap}
\usepackage[T1]{fontenc}
\usepackage{amsmath,amssymb,amstext}
\usepackage{babel}
\usepackage{times}
\usepackage[Bjarne]{fncychap}
\usepackage[dontkeepoldnames]{sphinx}

\usepackage{geometry}

% Include hyperref last.
\usepackage{hyperref}
% Fix anchor placement for figures with captions.
\usepackage{hypcap}% it must be loaded after hyperref.
% Set up styles of URL: it should be placed after hyperref.
\urlstyle{same}

\addto\captionsenglish{\renewcommand{\figurename}{Fig.}}
\addto\captionsenglish{\renewcommand{\tablename}{Table}}
\addto\captionsenglish{\renewcommand{\literalblockname}{Listing}}

\addto\captionsenglish{\renewcommand{\literalblockcontinuedname}{continued from previous page}}
\addto\captionsenglish{\renewcommand{\literalblockcontinuesname}{continues on next page}}

\addto\extrasenglish{\def\pageautorefname{page}}



\usepackage{enumitem}

\title{Loom Concepts and User Help Guide}
\date{Apr 07, 2018}
\release{0.7.0}
\author{Veritas Technologies LLC}
\newcommand{\sphinxlogo}{\sphinxincludegraphics{loom.png}\par}
\renewcommand{\releasename}{Release}
\makeindex

\begin{document}

\maketitle
\sphinxtableofcontents
\phantomsection\label{\detokenize{mcdmp_concepts::doc}}



\chapter{Loom Overview}
\label{\detokenize{mcdmp_concepts:content-mcdmp-concepts}}\label{\detokenize{mcdmp_concepts:loom-overview}}\label{\detokenize{mcdmp_concepts:loom-concepts-glossary}}
Veritas Multi-Cloud Data Management Platform (Loom) enables enterprises to exercise complete control over their myriad data assets, in a unified, flexible, scalable and highly available manner, irrespective of data location or application deployment model; on premise, cloud, or hybrid.

\begin{figure}[htbp]
\centering
\capstart

\noindent\sphinxincludegraphics[scale=0.75]{{_static/SS-GSG-Overview-MCDMP}.png}
\caption{Figure: Overview of Loom}\label{\detokenize{mcdmp_concepts:id1}}\end{figure}


\chapter{Loom Value}
\label{\detokenize{mcdmp_concepts:loom-value}}
Loom is designed as a platform that seamlessly plugs in storage and data management technologies from Veritas and other 3rd parties in future. It exposes intelligent information management functionality through contemporary REST interfaces. Thus, it not only speeds up creation of new information management services by Veritas but also enables business specific custom integrations and other 3rd party information management applications, on top of this platform.

The ‘Unified User-Experience’ provided by Loom platform’s ‘Single Point of Control’ helps CIOs to gain deep insights and better control of their business data.  Closed loop enterprise policy based information management offered by Loom works across business workloads and data assets owned by the enterprise.   Intelligent interfaces of Loom can accelerate development of new information driven solutions by DevOps and Data Scientists that can effectively address enterprise needs for information governance and agility in monetizing newer information insights. With these overarching objectives, Loom is poised to be a game changer in today’s digital economy driven by information.


\chapter{Why Loom?}
\label{\detokenize{mcdmp_concepts:why-loom}}
Digital economy has resulted in unprecedented growth of data, especially user data. Enterprises do not yet have complete clarity on fast growing unstructured dark data in their possession. Globally, there has been an increase in user awareness about privacy rights. Newer digital regulations pertaining to data ownership, retention, privacy and compliance entail enterprises to exactly know and establish a trace on what data they possess, where it resides within their business domains and how it is accessed, managed, stored and discarded.  What is needed is beyond current storage and data management solutions in order to address enterprise information governance needs.

\begin{figure}[htbp]
\centering
\capstart

\noindent\sphinxincludegraphics[scale=0.75]{{mcdmp-3-key-concepts}.png}
\caption{Figure: Storage, Data and Information Management Functions}\label{\detokenize{mcdmp_concepts:id2}}\end{figure}

So what is information governance? Enterprise Information Governance, as defined by Gartner, refers to specification of decision rights and an accountability framework to encourage desirable behavior in the valuation, creation, storage, use, archival and deletion of information. It includes the processes, roles, standards and metrics that ensure the effective and efficient use of information in enabling an organization to achieve its goals. Veritas Loom is aimed at Enterprise Information Governance to help businesses determine the two fundamental concerns:
\#. To which extent are the current business processes compliant with regulation standards such as GDPR, PII, others.
\#. Which goals do they need to achieve in order to be compliant with regulatory obligations.

Loom can help businesses move up the information governance value chain from simply protecting data to data orchestration, governance and eventually being able to monetize enterprise digital assets.

\begin{figure}[htbp]
\centering
\capstart

\noindent\sphinxincludegraphics[scale=0.75]{{_static/mcdmp-protect-to-monetize-value}.png}
\caption{Figure: Loom and Information Governance Maturity}\label{\detokenize{mcdmp_concepts:id3}}\end{figure}

Besides regulations, there is another factor that necessitates a solution such as Loom for today’s highly digital enterprises. Digital economy is fueling illiteracy of digital assets at a central level within an organization. There is unavailability of a complete information map that annotates information assets. It could potentially lead to lost business opportunities and failure to retrieve, optimize and monetize information insights.

Today, Veritas has several market leading point products that address information management including data storage, data archival and data management. Loom offers a platform that addresses modern enterprise need for greater insights into its data assets from a single unified interface \textendash{} Information Control Dashboard.  The platform enables enterprises to cut across storage and data management silos with a unified information management through a single pane of glass. Loom vision is to eventually leverage and offer all information and data management technologies developed by Veritas as well as 3rd party ones in future, with clean and neatly defined interfaces.


\chapter{Loom: Key Differentiators}
\label{\detokenize{mcdmp_concepts:loom-key-differentiators}}\begin{itemize}
\item {} 
Unstructured data, Scale, HA, FT, Reliability - Making sense of huge amounts of Unstructured DataUnstructured data, Scale, HA, FT, Reliability - Making sense of huge amounts of Unstructured Data

\item {} 
Closed Loop, Single Pane of glass unified interface

\item {} 
Multi-cloud support: Azure, Oracle, others (TBD)

\item {} 
Supports plug in of data sources belonging to application deployed using various deployment models such as bare-metal / VM / on premise / private / public / hybrid cloud

\item {} 
REST APIs for DevOps, Data Scientists to create new solutions on top of Loom

\item {} 
Automation Pipeline (Testing, certified to run on platform \textendash{} license, FED certification simplified)

\item {} 
Information Management presence across entire data life cycle

\item {} 
Seamless integration for offering Veritas Information Management Suite, Newer applications (3rd Party as well in future)

\end{itemize}


\chapter{How Loom works?}
\label{\detokenize{mcdmp_concepts:how-loom-works}}
In this section, we will cover how Loom works in more detail than described in the Getting Started Guide.  The focus here is architectural components, their interactions in the context of end user objectives such as visibility, GDPR etc. In terms of architecture, this would be a 10,000 feet view, as needed by end user, not by an engineer who is building Loom components. Also, we will describe Loom deployment options and how end users interact with the platform for information governance objectives.


\section{Loom High Level Architectural Components}
\label{\detokenize{mcdmp_concepts:loom-high-level-architectural-components}}
Loom is crafted using micro services based architecture.  These micro services operate across several layers within the platform:
\begin{itemize}
\item {} 
Control Plane

\item {} 
Data Plane

\item {} 
Analytics Plane

\item {} 
Policy based closed-loop self-healing remedial action automation engine

\item {} 
System monitoring, logging, trace functionalities

\end{itemize}

\begin{figure}[htbp]
\centering
\capstart

\noindent\sphinxincludegraphics[scale=0.75]{{mcdmp-arch-planes}.png}
\caption{Figure: Components of Loom}\label{\detokenize{mcdmp_concepts:id4}}\end{figure}

Loom micro services work in tandem, together, to cater to the following Information Governance needs in an enterprise:
\begin{itemize}
\item {} 
discovery,

\item {} 
visualization,

\item {} 
classification,

\item {} 
secured data asset management, and

\item {} 
compliance and regulatory requirements.

\end{itemize}

Besides these, enterprises can take remedial actions for information compliance and governance through the policy engine.  Alerts and warnings in the MDCMP Dashboard can provide valuable insights related to information policy breaches in a timely manner.
These micro services are flexible, multi-tenant, highly available, scalable and fault tolerant. They rely heavily on latest proven enterprise grade technologies such as Dockers, Kubernetes and Clustering.


\section{Loom Deployment Options}
\label{\detokenize{mcdmp_concepts:loom-deployment-options}}
Enterprises can choose to deploy Loom based on their business and IT requirements from various deployment options available to users.
Loom can be deployed on clusters running in:
\begin{itemize}
\item {} 
virtualized data-centers,

\item {} 
private cloud environment,

\item {} 
infrastructure offered by any supported public cloud IaaS/PaaS environment -
* AWS
* Azure
* Oracle
* Others (TBD)

\end{itemize}

\begin{figure}[htbp]
\centering
\capstart

\noindent\sphinxincludegraphics[scale=0.75]{{mcdmp-deployment-options}.png}
\caption{Figure: Loom Deployment Models}\label{\detokenize{mcdmp_concepts:id5}}\end{figure}

Customers can deploy core Loom components along with add-on services that run on top of the platform.  These add-on services offer capabilities such as information security, discovery, visualization, compliance \& GDPR assessment.


\section{Loom Usage Model}
\label{\detokenize{mcdmp_concepts:loom-usage-model}}
End users typically interact with the Loom via either of the two ways:
* Unified Dashboard,
* REST based API SDK

The Dashboard and APIs offer information management features through an intuitive, single pane of glass, easy to use model instead of having to interact with several information management point products.

\sphinxstylestrong{Information Map Usage Model}
An enterprise Data Officers (DO) may need to obtain a high level information map to figure out whether and how many copies of sensitive or regulated information is stored globally across the enterprise data assets.  To achieve this, they need to have Loom deployed and users accounts set up by Loom Admin.  For details refer to MDCMP Administration Guide. Next, user need to log into the Dashboard ( or in case of API access, perform appropriate authentication and authorization steps).
After successful log in, user needs to manually configure (or scan and auto-discover) various information sources within an organization and connect them to the data plane.  \{ placeholder note: What is the connection between DO/user, enterprise, and data plane (1:1 or 1:many) \textendash{} needs to be validated by Loom team before adding here.\}  The information map application runs on top of the platform and an authenticated user can switch to the map view using sidebar and see various information sources, locations and their attributes via the Dashboard.  The following insights can be gleaned via information map:
\begin{itemize}
\item {} 
Location of Data sources

\item {} 
Events / Breaches / Copies of data

\item {} 
Others

\end{itemize}

\sphinxstylestrong{Information Classification Usage Model}
A DO can classify information across the enterprise using Loom 360 data management application.  More details TBD

\sphinxstylestrong{Information Governance Usage Model}
TBD


\chapter{Loom Terminology}
\label{\detokenize{mcdmp_concepts:loom-terminology}}
\begin{sphinxShadowBox}
\sphinxstyletopictitle{Glossary}
\begin{itemize}
\item {} 
\phantomsection\label{\detokenize{mcdmp_concepts:id23}}{\hyperref[\detokenize{mcdmp_concepts:control-and-data-planes}]{\sphinxcrossref{Control And Data Planes}}}

\item {} 
\phantomsection\label{\detokenize{mcdmp_concepts:id24}}{\hyperref[\detokenize{mcdmp_concepts:information-asset}]{\sphinxcrossref{Information Asset}}}

\item {} 
\phantomsection\label{\detokenize{mcdmp_concepts:id25}}{\hyperref[\detokenize{mcdmp_concepts:information-management}]{\sphinxcrossref{Information Management}}}

\item {} 
\phantomsection\label{\detokenize{mcdmp_concepts:id26}}{\hyperref[\detokenize{mcdmp_concepts:data-management}]{\sphinxcrossref{Data Management}}}

\item {} 
\phantomsection\label{\detokenize{mcdmp_concepts:id27}}{\hyperref[\detokenize{mcdmp_concepts:storage-management}]{\sphinxcrossref{Storage Management}}}

\item {} 
\phantomsection\label{\detokenize{mcdmp_concepts:id28}}{\hyperref[\detokenize{mcdmp_concepts:loom-application}]{\sphinxcrossref{Loom Application}}}

\item {} 
\phantomsection\label{\detokenize{mcdmp_concepts:id29}}{\hyperref[\detokenize{mcdmp_concepts:loom-application-models}]{\sphinxcrossref{Loom Application Models}}}

\item {} 
\phantomsection\label{\detokenize{mcdmp_concepts:id30}}{\hyperref[\detokenize{mcdmp_concepts:loom-solution}]{\sphinxcrossref{Loom Solution}}}

\item {} 
\phantomsection\label{\detokenize{mcdmp_concepts:id31}}{\hyperref[\detokenize{mcdmp_concepts:workload}]{\sphinxcrossref{Workload}}}

\item {} 
\phantomsection\label{\detokenize{mcdmp_concepts:id32}}{\hyperref[\detokenize{mcdmp_concepts:application-manager}]{\sphinxcrossref{Application Manager}}}

\item {} 
\phantomsection\label{\detokenize{mcdmp_concepts:id33}}{\hyperref[\detokenize{mcdmp_concepts:loom-services}]{\sphinxcrossref{Loom  Services}}}

\item {} 
\phantomsection\label{\detokenize{mcdmp_concepts:id34}}{\hyperref[\detokenize{mcdmp_concepts:loom-roles}]{\sphinxcrossref{Loom Roles}}}

\item {} 
\phantomsection\label{\detokenize{mcdmp_concepts:id35}}{\hyperref[\detokenize{mcdmp_concepts:loom-organizations}]{\sphinxcrossref{Loom Organizations}}}

\item {} 
\phantomsection\label{\detokenize{mcdmp_concepts:id36}}{\hyperref[\detokenize{mcdmp_concepts:loom-policy}]{\sphinxcrossref{Loom Policy}}}

\item {} 
\phantomsection\label{\detokenize{mcdmp_concepts:id37}}{\hyperref[\detokenize{mcdmp_concepts:loom-policy-rule-sets}]{\sphinxcrossref{Loom Policy Rule Sets}}}

\item {} 
\phantomsection\label{\detokenize{mcdmp_concepts:id38}}{\hyperref[\detokenize{mcdmp_concepts:loom-asset}]{\sphinxcrossref{Loom Asset}}}

\item {} 
\phantomsection\label{\detokenize{mcdmp_concepts:id39}}{\hyperref[\detokenize{mcdmp_concepts:loom-data-connectors}]{\sphinxcrossref{Loom Data Connectors}}}

\item {} 
\phantomsection\label{\detokenize{mcdmp_concepts:id40}}{\hyperref[\detokenize{mcdmp_concepts:persistent-pod-ppod}]{\sphinxcrossref{Persistent Pod (PPOD)}}}

\item {} 
\phantomsection\label{\detokenize{mcdmp_concepts:id41}}{\hyperref[\detokenize{mcdmp_concepts:classification}]{\sphinxcrossref{Classification}}}

\item {} 
\phantomsection\label{\detokenize{mcdmp_concepts:id42}}{\hyperref[\detokenize{mcdmp_concepts:loom-pod}]{\sphinxcrossref{Loom Pod}}}

\item {} 
\phantomsection\label{\detokenize{mcdmp_concepts:id43}}{\hyperref[\detokenize{mcdmp_concepts:loom-role-based-access-control-rbac}]{\sphinxcrossref{Loom Role Based Access Control (RBAC)}}}

\item {} 
\phantomsection\label{\detokenize{mcdmp_concepts:id44}}{\hyperref[\detokenize{mcdmp_concepts:loom-job}]{\sphinxcrossref{Loom Job}}}

\item {} 
\phantomsection\label{\detokenize{mcdmp_concepts:id45}}{\hyperref[\detokenize{mcdmp_concepts:loom-gdpr-terminology}]{\sphinxcrossref{Loom GDPR Terminology}}}

\end{itemize}
\end{sphinxShadowBox}


\section{Control And Data Planes}
\label{\detokenize{mcdmp_concepts:control-and-data-planes}}\label{\detokenize{mcdmp_concepts:def-term-planes}}
A key architectural facet of Loom Platform is the concept of two separate planes - Control and Data Planes. This approach results in significant performance, quality of service (QoS), resiliency, scalability and cost benefits for multi-cloud data management functions offered by Loom Platform.  Multi-tenancy is yet another tenet of Loom Platform architecture. It is supported by Loom Platform inherently at all levels, including the Control Plane and Data Planes.

There is a single Control Plane and one or more data planes in a typical MDCM Platform deployment. Both Control and Data Planes comprise of a bunch of micro-services that cater to functionalities such as Logging, Monitoring, Asset Database and Data Pipelining.  Control Plane is geared towards Loom workflow management and orchestration whereas Data Planes are primarily focused around data source connections and data pipelining aspects.

\begin{figure}[htbp]
\centering
\capstart

\noindent\sphinxincludegraphics[scale=0.75]{{control-many-data-planes}.png}
\caption{Figure: Loom Control \& Data Planes}\label{\detokenize{mcdmp_concepts:id6}}\end{figure}

Loom Platform architecture comprises of a set of microservices that cater to core platform functions, data ingestion functions, asset management functions, and application functions. These micro-services work across control and data planes.

Core Platform services include:
\begin{itemize}
\item {} 
Application Manager,

\item {} 
Workflow Management,

\item {} 
Job Management,

\item {} 
User Access Management,

\item {} 
Monitoring

\item {} 
Logging

\item {} 
Auditing

\item {} 
Data Encryption

\item {} 
Search

\end{itemize}

Data Ingestion services include:
\begin{itemize}
\item {} 
Distributed Streaming

\item {} 
Object Browser

\end{itemize}

Asset Management services include:
\begin{itemize}
\item {} 
AssetDB Service

\item {} 
Data Graphs

\end{itemize}

Application services include:
\begin{itemize}
\item {} 
Data Source Connection Management

\item {} 
Copy Management

\item {} 
Data Classification

\item {} 
Data Discovery

\item {} 
Policy Manager

\end{itemize}

The figure below shows a high level architecture diagram visualizing how Loom supports multi-tenant customer environment via Control Plane, Data Planes and various micro-services that help end users harness the enterprise data sources for information insights.

\begin{figure}[htbp]
\centering
\capstart

\noindent\sphinxincludegraphics[scale=0.75]{{_static/mcdmp-planes}.png}
\caption{Figure: Loom’s Multi-tenant architecture: Using Control \& Data Planes, micro-services to harness Data Sources}\label{\detokenize{mcdmp_concepts:cpt-planes-img}}\label{\detokenize{mcdmp_concepts:id7}}\end{figure}


\section{Information Asset}
\label{\detokenize{mcdmp_concepts:information-asset}}\label{\detokenize{mcdmp_concepts:term-info-asset}}
An Information Asset is the application information, content or both that has monetary value for the business user.  An information asset can also be a container of other such assets. For e.g., Data center, IaaS platform, Computer Storage, Virtual Machine, File-System, Cloud Service etc.

In other words, an \sphinxstylestrong{Information Asset} refers to any source of information for an enterprise.  For example, it could be a specific data storage device, a shared CIFS or NFS drive, cloud based object storage, a database or a collection of unstructured data, archived data, backed up data or any enterprise owned data stored on an accessible digital media.

\begin{figure}[htbp]
\centering
\capstart

\noindent\sphinxincludegraphics[scale=0.75]{{info_asset}.png}
\caption{Figure: Information Asset}\label{\detokenize{mcdmp_concepts:id8}}\end{figure}


\section{Information Management}
\label{\detokenize{mcdmp_concepts:information-management}}
Information Management (IM) refers to acquisition of information from one or more sources, the custodianship and the distribution of that information to those who need it, and its ultimate disposition through archiving or deletion. Loom Platform enables enterprises to achieve the following information management functions:
\begin{itemize}
\item {} 
Information Insights

\item {} 
Compliance

\item {} 
Governance

\item {} 
Monetization

\end{itemize}

Loom Platform helps classify information across the enterprise and discover \sphinxstylestrong{Data Locality}. This enables informed decision making about \sphinxstylestrong{Data Retention}, which data is important and needs to be retained and archived.  Besides these, Loom Platform also enables another aspect of information management - data access security and control.

\begin{figure}[htbp]
\centering
\capstart

\noindent\sphinxincludegraphics[scale=0.75]{{info_mngt}.png}
\caption{Figure: Information Management}\label{\detokenize{mcdmp_concepts:id9}}\end{figure}


\section{Data Management}
\label{\detokenize{mcdmp_concepts:data-management}}
Enterprise Data Management is a “framework” comprising of processes, policies and services that ensure that mission-critical data assets are formally managed and aligned throughout the enterprise to achieve a single version of the truth.  This includes:
\begin{itemize}
\item {} 
Data Visualization

\item {} 
Disaster Recovery and High Availability

\item {} 
Backup

\item {} 
Data Migration

\item {} 
Copy Life Cycle Management

\end{itemize}

\begin{figure}[htbp]
\centering
\capstart

\noindent\sphinxincludegraphics[scale=0.75]{{data_mngt}.png}
\caption{Figure: Data Management}\label{\detokenize{mcdmp_concepts:id10}}\end{figure}


\section{Storage Management}
\label{\detokenize{mcdmp_concepts:storage-management}}
Storage Management deals with facilitating enterprise data storage, provisioning, automation and virtualization. It comprises of the following functionalities:
\begin{itemize}
\item {} 
Storage asset discovery and visibility

\item {} 
Secured Access

\item {} 
SLA (Placement, Cost, Performance)

\item {} 
Software Defined Storage (SDS, Data Model, Virtualization)

\end{itemize}

\begin{figure}[htbp]
\centering
\capstart

\noindent\sphinxincludegraphics[scale=0.75]{{stg_mngt}.png}
\caption{Figure: Storage Management}\label{\detokenize{mcdmp_concepts:id11}}\end{figure}


\section{Loom Application}
\label{\detokenize{mcdmp_concepts:loom-application}}\label{\detokenize{mcdmp_concepts:term-application}}
Loom is a Digital Information Management Platform. It comprises of a set of core micro-services that enable structured and unstructured data insight gathering and analysis in a highly scalable manner. Besides these micro-services, the Loom Ecosystem also consists of Loom Applications which are deployed on top of the core Loom Platform and utilize the Loom micro-services for processing data.

The term Loom ‘Application’ refers to any application such as DataVision, GDPR, and others in future that are built using Loom REST interfaces and deployed to run on top of the Loom. These Applications are to address different enterprise business imperatives, strategic objectives and budgeted initiatives.  In future, when Loom Platform SDK is available for external use, enterprises could develop 3rd party applications to address their specific needs.

A Loom Application represents business capabilities to solve one or more customer pain points or to deliver a specific business outcome.  A business capability is the unit of delivery (or license or subscription) upon which market value is projected and to which revenue is attached. It is also the unit of “Try \& Buy”, telemetry and customer feedback exposed through business function APIs served by Loom.

\begin{figure}[htbp]
\centering
\capstart

\noindent\sphinxincludegraphics[scale=0.75]{{access_mcdmp_apps}.png}
\caption{Figure: Loom Applications}\label{\detokenize{mcdmp_concepts:id12}}\end{figure}


\section{Loom Application Models}
\label{\detokenize{mcdmp_concepts:term-loom-app-models}}\label{\detokenize{mcdmp_concepts:loom-application-models}}
In principle, there are three basic models of Loom Applications:
\begin{enumerate}
\item {} 
\sphinxstylestrong{Model 1}: Application running on top of the Loom platform as a suite of micro services such as UX, business logic and value add services. This model of applications run as part of Loom Run Time Environment (RTE).  These applications not only have access to the Loom platform services and infrastructure, but also share resources, security domain and multi-cloud connectivity of Loom core services themselves.

\item {} 
\sphinxstylestrong{Model 2}: Applications running on the same provider (PaaS, IaaS) instance that the Loom core services run on.  In this model, application is deployed and operated independently of Loom. Integration with Loom is exclusively through the Loom APIs and no platform resources or services are shared. In this case Application runs outside Loom Security Domain.  However, the application indirectly shares PaaS and IaaS resources which Loom too is using in this case.

\item {} 
\sphinxstylestrong{Model 3}: Standalone Applications model comprises of proprietary, independent 3rd party applications that are deployed, run and operated completely independent of the Loom instance.  No resources are shared across the platform and the application in this case. Integration with Loom is exclusively through Loom APIs.

\end{enumerate}

For Veritas provided applications (Info-visibility, GDPR, Cloud-Data-Protection, etc) model (1) is the desired Veritas Application model.

\begin{figure}[htbp]
\centering
\capstart

\noindent\sphinxincludegraphics[scale=0.75]{{mcdmp_app_model}.png}
\caption{Figure: Loom Application Model}\label{\detokenize{mcdmp_concepts:id13}}\end{figure}


\section{Loom Solution}
\label{\detokenize{mcdmp_concepts:loom-solution}}
The term ‘Loom Solution’ refers to a set of applications that are built using Loom REST interfaces and are deployed to run on top of the Loom platform as a bundle to solve a particular enterprise pain point - such as Information Governance, Information Visibility and gaining insights into enterprise dark data.

\begin{figure}[htbp]
\centering
\capstart

\noindent\sphinxincludegraphics[scale=0.75]{{mcdmp_apps}.png}
\caption{Figure: Loom Solution}\label{\detokenize{mcdmp_concepts:id14}}\end{figure}


\section{Workload}
\label{\detokenize{mcdmp_concepts:workload}}
The term ‘Workload’ refers to third-party multi-tier applications deployed at enterprises such as Web Servers, Oracle Enterprise Server, PostgresQL, others.  These workloads utilize enterprise IT resources (on premise or in the cloud) and access enterprise information / data assets.  Newer enterprise data assets and information is generated as well via these workloads.

\begin{figure}[htbp]
\centering
\capstart

\noindent\sphinxincludegraphics[scale=0.75]{{workload}.png}
\caption{Figure: Workload}\label{\detokenize{mcdmp_concepts:id15}}\end{figure}


\section{Application Manager}
\label{\detokenize{mcdmp_concepts:application-manager}}
Application Manager is one of the key components of Loom Platform. It runs in the Control Plane and manages life cycle of all Loom applications deployed on the Platform.  Besides this, Application Manager also provides interfaces (programmatic, UX based) to query attributes of these applications such as version, associated resources etc.

Typical Application Life-cycle related activities comprise of:
\begin{itemize}
\item {} 
Installation

\item {} 
UnInstallation

\item {} 
Application Upgrades

\item {} 
Application Rollback

\item {} 
Keeping track of Application Status, resource usage etc.

\end{itemize}

\begin{figure}[htbp]
\centering
\capstart

\noindent\sphinxincludegraphics[scale=0.75]{{mcdmp_apps}.png}
\caption{Figure: Loom Application Manager}\label{\detokenize{mcdmp_concepts:id16}}\end{figure}


\section{Loom  Services}
\label{\detokenize{mcdmp_concepts:loom-services}}
There are several micro services that make up the Veritas Loom core platform ecosystem.  Some of these services are internal to Loom and not exposed to Loom Users.


\sphinxstrong{See also:}


This is a \sphinxstylestrong{DOC-CONTRIB} Request

Need PO / Anand’s help to get the facts correct here.
The following information is based on the confluence page \sphinxhref{https://jira.community.veritas.com/browse/IMP-3695}{Jira 3695} and the email listed in Loom Issue \sphinxhref{https://confluence.community.veritas.com/pages/viewpage.action?pageId=129341862}{Loom Confluence\_129341862}



The table below lists various Loom services.


\begin{savenotes}\sphinxattablestart
\centering
\begin{tabulary}{\linewidth}[t]{|T|T|T|}
\hline
\sphinxstylethead{\sphinxstyletheadfamily 
Loom Service
\unskip}\relax &\sphinxstylethead{\sphinxstyletheadfamily 
Type
\unskip}\relax &\sphinxstylethead{\sphinxstyletheadfamily 
Purpose
\unskip}\relax \\
\hline
Airflow Service
&
\raisebox{-0.5\height}{\sphinxincludegraphics[scale=0.75]{{internal-svc}.png}}
&
TBD
\\
\hline
Application Manager
&
\raisebox{-0.5\height}{\sphinxincludegraphics[scale=0.75]{{external-svc}.png}}
&
Application Manager manages life cycle of Loom applications
installed as part of Loom deployment.
\\
\hline
Cluster Manager
&
\raisebox{-0.5\height}{\sphinxincludegraphics[scale=0.75]{{internal-svc}.png}}
&
Loom Infrastructure Management Service
\\
\hline
Job Management
&
\raisebox{-0.5\height}{\sphinxincludegraphics[scale=0.75]{{external-svc}.png}}
&
Handles Loom Job scheduling and tracking. Loom Job could be
data classification job for a specified content source.
\\
\hline
AssetDB Service
&
\raisebox{-0.5\height}{\sphinxincludegraphics[scale=0.75]{{internal-svc}.png}}
&
TBD
\\
\hline
Data Pipeline
&
\raisebox{-0.5\height}{\sphinxincludegraphics[scale=0.75]{{internal-svc}.png}}
&
TBD
\\
\hline
Data Streaming
&
\raisebox{-0.5\height}{\sphinxincludegraphics[scale=0.75]{{internal-svc}.png}}
&
TBD
\\
\hline
Loom Monitoring
&
\raisebox{-0.5\height}{\sphinxincludegraphics[scale=0.75]{{internal-svc}.png}}
&
TBD
\\
\hline
Object Browser
&
\raisebox{-0.5\height}{\sphinxincludegraphics[scale=0.75]{{internal-svc}.png}}
&
TBD
\\
\hline
Identity Management
&
\raisebox{-0.5\height}{\sphinxincludegraphics[scale=0.75]{{internal-svc}.png}}
&
TBD
\\
\hline
Secret Management
&
\raisebox{-0.5\height}{\sphinxincludegraphics[scale=0.75]{{internal-svc}.png}}
&
TBD
\\
\hline
Loom UI Service
&
\raisebox{-0.5\height}{\sphinxincludegraphics[scale=0.75]{{external-svc}.png}}
&
TBD
\\
\hline
Connector Service
&
\raisebox{-0.5\height}{\sphinxincludegraphics[scale=0.75]{{internal-svc}.png}}
&
TBD
\\
\hline
Data Plane Producer
&
\raisebox{-0.5\height}{\sphinxincludegraphics[scale=0.75]{{internal-svc}.png}}
&
TBD
\\
\hline
Scheduler Service
&
\raisebox{-0.5\height}{\sphinxincludegraphics[scale=0.75]{{internal-svc}.png}}
&
TBD
\\
\hline
Data Classification
&
\raisebox{-0.5\height}{\sphinxincludegraphics[scale=0.75]{{internal-svc}.png}}
&
TBD
\\
\hline
Policy Manager
&
\raisebox{-0.5\height}{\sphinxincludegraphics[scale=0.75]{{external-svc}.png}}
&
TBD
\\
\hline
Asset Manager
&
\raisebox{-0.5\height}{\sphinxincludegraphics[scale=0.75]{{internal-svc}.png}}
&
TBD
\\
\hline
Workflow Manager
&
\raisebox{-0.5\height}{\sphinxincludegraphics[scale=0.75]{{internal-svc}.png}}
&
TBD
\\
\hline
Others?
&
\raisebox{-0.5\height}{\sphinxincludegraphics[scale=0.75]{{internal-svc}.png}}
&
TBD
\\
\hline
\end{tabulary}
\par
\sphinxattableend\end{savenotes}


\section{Loom Roles}
\label{\detokenize{mcdmp_concepts:content-loom-roles}}\label{\detokenize{mcdmp_concepts:loom-roles}}
In the context of Loom, there can be five types of user roles:
\begin{itemize}
\item {} 
Loom Admin

\item {} 
Partner Admin

\item {} 
Customer Admin

\item {} 
Tenant Admin

\item {} 
Tenant User

\end{itemize}

Depending upon the kind of Loom deployment, one or more roles may not be available.  For example, in case of Loom Alpha1, only Azure based Loom SaaS deployment is supported.  In this case, only Customer Admin, Tenant Admin and Tenant User roles exist.

Loom Administration role is split into several entities as appropriate for a deployment type and enterprise requirements for information management.  In case of Cloud based SaaS deployment (for e.g. Azure), Veritas or its Partner would set up the SaaS infrastructure to deploy and bootstrap core components of Loom. This role is referred to as Loom Admin.  Other administrative actions of Loom Admin include setting up of the Customer Admin and Tenant Admin accounts.  The Customer Admin role represents an organization or enterprise. This role is tasked with creation and deletion of Tenant Admins and Tenant Users.

The Tenant Administrators are in charge of bringing up their organization specific Data Engines. These Data Engines are set up to use the Veritas Data Connector framework that binds various enterprise data sources (CIFS, NFS, Box, others ) to the Loom deployment in the context of an organization and enables the the end users to access, visualize and govern enterprise data assets. The Tenant Admin could further delegate the Data Source bindings to a Storage or IT Admin within their organizations by creating additional users with Tenant Admin privileges.  Besides the Storage, the Tenant Admin also manage the Loom Applications such as 360 Data Analysis,  GDPR (not available in Alpha1) and others that are deployed as part of Loom subscription. The Tenant Admin is entrusted with creation and deletion of Tenant user accounts for their organization and setting up policies related to data access, classification and visibility which are applicable enterprise wide. The Policy feature may be limited in scope for Loom Alpha1 release.

\sphinxstylestrong{Loom Admin:} Special Loom user privilege, the very first Loom user responsible for deploying, setting up and administrating Loom instance. Loom Admin has privileges for creating Customer Administrators and deleting them when required.

\sphinxstylestrong{Partner Admin:} This is a special role that can be considered as a sibling of Loom Admin to support Veritas Partners that may want to onboard several tenants on the Loom deployment.  For example, a MSP or Cloud Provider could be a Veritas Partner.  A Partner Admin can create one or more Tenant Admin users as needed.

\begin{sphinxadmonition}{note}{Note:}
Loom Alpha1 Related Information

Loom Admin role is not exposed to any other users as the deployment is managed by Veritas. Also, there is no Partner Admin role supported in Alpha1 release.
\end{sphinxadmonition}

\sphinxstylestrong{Customer Admin:} As part of Loom trial subscription, each organization gets its own Loom Account which has Customer Admin privileges. This role is assigned with user management for the enterprise.  Customer Admins can create Tenant Admins for two key functions - Tenant User Management and to deploy organization specific Loom Applications.  Not all organizations may use the same set of applications.

\sphinxstylestrong{Tenant Admin:} Delegated, specialized role, created by Customer Admin during the initial stages of onboarding Loom. It has privileges to create Tenant users and perform other administrator tasks in the context of a Tenant such as IT or Storage administrator who binds enterprise data assets to the Loom Data Plane through credential / access rights. The Tenant Admin can create end users for only their organization . Another function performed by Tenant Admin is defining and setting up enterprise / customer policies for data scans, classification, access and regulatory compliance. This feature may not be fully supported in Loom Alpha1 release.

\sphinxstylestrong{Tenant Users:} These are typical Loom end user accounts belonging to a particular organization that are responsible for enterprise information governance. These users will carry out various tasks on the system including monitoring enterprise data assets, conducting audits corresponding to a legal inquiry, retrieving data access or data trail reports etc.


\section{Loom Organizations}
\label{\detokenize{mcdmp_concepts:term-organizations}}\label{\detokenize{mcdmp_concepts:loom-organizations}}
The term ‘Organizations’ refers to enterprises or customers that wish to subscribe to Loom SaaS deployment on Microsoft Azure. Figure below shows the ‘Organizations’ entry in the left hand side navigation bar.  This entry is visible only to Loom Platform Admin or Loom Partner Admin who are allowed to create new customer accounts.

TBD - We need more conceptual, Loom usage and Loom Help related content for Loom topic - Organizations.

\begin{figure}[htbp]
\centering
\capstart

\noindent\sphinxincludegraphics[scale=0.75]{{loom-organizations}.png}
\caption{Figure: Loom Organizations Navigation Bar Entry}\label{\detokenize{mcdmp_concepts:id17}}\end{figure}


\section{Loom Policy}
\label{\detokenize{mcdmp_concepts:loom-policy}}\label{\detokenize{mcdmp_concepts:term-loom-policy}}
\sphinxstylestrong{Policy Definition}

Loom Policy is a collection of rule sets. Each Loom policy definition can be associated with multiple {\hyperref[\detokenize{mcdmp_concepts:term-loom-asset}]{\sphinxcrossref{\DUrole{std,std-ref}{Loom Asset}}}} or {\hyperref[\detokenize{mcdmp_concepts:term-loom-connectors}]{\sphinxcrossref{\DUrole{std,std-ref}{Loom Data Connectors}}}}.

\sphinxstylestrong{Policy Type}

Each policy has an associated ‘type’ which denotes what kinds of Loom Action, assets, connectors are supported by the policy.

\sphinxstylestrong{Rule Sets}

Rule Sets are collections of rules (conditions) that are associated with one or more of the enterprise content repositories or data sources.

\sphinxstylestrong{Policy Instance}

Policy definition combined with an asset and a specified policy execution schedule results in a “Policy Instance”. This is also referred to as Policy Binding.

\sphinxstylestrong{Default Policy}

It is a special type of policy which automatically gets bound to an asset when the asset gets added in the system. A default policy has default schedule.


\section{Loom Policy Rule Sets}
\label{\detokenize{mcdmp_concepts:term-policy-rule-sets}}\label{\detokenize{mcdmp_concepts:loom-policy-rule-sets}}
Policy rule sets are essentially conditions that define pattern matches that an application looks for when executing actions on connectors. A rule set specifies the criteria that an item must meet for an application to consider it a match. Your policies can contain any number of conditions.

Currently the Loom platform allows you to create rules to include or exclude a pattern, file, or folder. By default, the application will discover or scan all data levels for all configured connectors without any exclusion. The Loom Classify policy currently has a default rule to exclude any file more than 10MB and any file with .bat and .exe extensions.


\subsection{Basic components of a rule set}
\label{\detokenize{mcdmp_concepts:basic-components-of-a-rule-set}}
All rule sets have this basic form:

property operator value

For example, in the following rule set, “size” is the property, “contains” is the operator, and “C\$” is the value:

The property specifies the type or characteristic of an item that you want to evaluate: whether a file, folder, or a pattern in the name or content of a connector.


\section{Loom Asset}
\label{\detokenize{mcdmp_concepts:term-loom-asset}}\label{\detokenize{mcdmp_concepts:loom-asset}}
The term ‘asset’ refers to a container that holds structured or unstructured data. For example, a file, a directory, a share, a filer.
TBD in the context of Loom.

TBD - Verify and update if a Loom Asset is the same as {\hyperref[\detokenize{mcdmp_concepts:term-info-asset}]{\sphinxcrossref{\DUrole{std,std-ref}{Information Asset}}}}. Require Loom Architecture team’s inputs regarding concept of Loom Asset here.


\section{Loom Data Connectors}
\label{\detokenize{mcdmp_concepts:term-loom-connectors}}\label{\detokenize{mcdmp_concepts:loom-data-connectors}}
The term ‘Connector’ refers to an adapter which enables Loom to obtain metadata and other insights from enterprise content repositories. Loom supports more than 20 different data connectors:
\begin{itemize}
\item {} 
CIFS Shares

\item {} 
NFS Shares

\item {} 
SharePoint Drive

\item {} 
SharePoint Online

\item {} 
OneDrive

\item {} 
AWS S3

\item {} 
Google Cloud Storage

\item {} 
Microsoft Azure Storage

\item {} 
Others TBD

\end{itemize}

More TBD. Need Loom Architect team’s inputs in this section.


\section{Persistent Pod (PPOD)}
\label{\detokenize{mcdmp_concepts:persistent-pod-ppod}}\label{\detokenize{mcdmp_concepts:term-ppod}}
The term ‘PPOD’ in Loom context refers to Persistent Pod which is a container that stores information about Loom Tenants in an encrypted manner.  The Loom architecture ensures scalability and high availability of these Persistent Pods. There are intelligent algorithms deployed (in the works at the moment) to determine which PPOD hosts the Loom Tenant information when a new Tenant is added to the Loom deployment in Azure SaaS. In future, when Loom supports on premises deployment, the same algorithm or its variation will ensure that persistent data for each tenant is cleanly isolated, fault tolerant and scalable.

There may be two or more PPODs in a typical Loom deployment, depending upon the number of Tenants, configuration setup for high availability etc.

Several factors govern which PPOD is selected when a new Tenant is introduced into the Loom ecosystem. Some of these factors are:
\begin{enumerate}
\item {} 
Load on the existing PPOD

\item {} 
Whether the new Tenant belongs to a pre-existing customer (Affinity based)

\item {} 
SLA requirements of a new Tenant

\end{enumerate}

\begin{figure}[htbp]
\centering
\capstart

\noindent\sphinxincludegraphics[scale=0.75]{{_static/term_ppod}.png}
\caption{Figure: Loom Persistent POD (PPOD)}\label{\detokenize{mcdmp_concepts:id18}}\end{figure}


\section{Classification}
\label{\detokenize{mcdmp_concepts:term-classification}}\label{\detokenize{mcdmp_concepts:classification}}
The term ‘Classification’ in Loom context refers to classifying customer data located in configured and connected enterprise content repositories. Data classification tags data according to its type, sensitivity, and value to the organization if altered, stolen, or destroyed. It helps an organization understand the value of its data, determine whether the data is at risk, and implement controls to mitigate risks.  Data classification also helps an organization comply with relevant industry-specific regulatory mandates such as SOX, HIPAA, PCI DSS, and GDPR. Veritas uses patent pending and machine learning enabled ‘Classification Engine’ technology to help customers discover, scan and understand data for better information management and monetization.

\begin{figure}[htbp]
\centering
\capstart

\noindent\sphinxincludegraphics[scale=1.0]{{_static/data_cls}.png}
\caption{Figure: Data Classification}\label{\detokenize{mcdmp_concepts:id19}}\end{figure}


\subsection{Classification Process}
\label{\detokenize{mcdmp_concepts:classification-process}}\label{\detokenize{mcdmp_concepts:term-cls-process}}
Following are the typical steps of enterprise data classification process:
\begin{enumerate}
\item {} 
\sphinxstylestrong{Data Discovery:} Determine the location, volume, and context of data on premises, in the cloud, structured or unstructured data within the enterprise asset landscape.

\item {} 
\sphinxstylestrong{Align Business needs:} Define data classification policies.

\item {} 
\sphinxstylestrong{Scan:} Dig deeper, execute classification based on tags, policies and analyze discovered data assets.

\item {} 
\sphinxstylestrong{Compliance:} Implement enforcement technologies to protect classified data (e.g., user access management, privileged user monitoring, sensitive data auditing,detecting information access breach, addressing compliance requirements such as GDPR)

\end{enumerate}

\begin{figure}[htbp]
\centering
\capstart

\noindent\sphinxincludegraphics[scale=1.0]{{gdpr_cls}.png}
\caption{Figure: Loom Data Classification for GDPR}\label{\detokenize{mcdmp_concepts:id20}}\end{figure}


\subsection{Native Classification}
\label{\detokenize{mcdmp_concepts:term-native-cls}}\label{\detokenize{mcdmp_concepts:native-classification}}
Native Classification refers to ‘data classification’ feature offered by the content repository provider - or a third party solution being used by an enterprise.  For example, data classification features offered by Box, OneDrive or Azure to tag and classify sensitive data or critical data within an organization.

More TBD - Need help from Loom Architects to define this accurately in the context of Loom.


\section{Loom Pod}
\label{\detokenize{mcdmp_concepts:loom-pod}}\label{\detokenize{mcdmp_concepts:term-loom-pod}}
The term ‘Pod’ in Loom context refers to an instance of one of the Loom micro-services hosted inside a container and isolated from other Loom services. Each Pod is a self-contained, smallest unit of deployment of Loom.  Loom comprises of several such micro-services, each of them deployed as a ‘Pod’.

Loom Architecture does not mandate (? TBV with architects) setting up each Loom micro-service in its own dedicated container. In the alpha release, this is how each Loom micro service is packaged for isolation, multi-tenancy support and scalability, fault tolerance objectives. One of the key building blocks of Loom is ‘Kubernetes’ and hence the ‘Pod’ terminology.

More on the concept of Loom Pod to be added as part of review with Loom Architects team.

To understand Loom Pod better, let us look at some of the Docker and Kubernetes basics such as Pods, Deployments and Services.
\begin{description}
\item[{\sphinxstylestrong{Pods}}] \leavevmode
Pods are the smallest deployable units in Kubernetes. A Pod consists of one or more containers and shared resources, such as data volumes and network addresses. Pods are tightly-coupled and act as an independent unit, much like containers themselves.

Similarly, a Loom Pod is the smallest deployable unit in the Loom Platform ecosystem.

\item[{\sphinxstylestrong{Deployments}}] \leavevmode
A Deployment governs how an instance of a \sphinxstylestrong{Pod} is created, deployed and maintained using a pre-defined configuration file. Deployments typically manage replication, scaling and restarting of the Pods.

\item[{\sphinxstylestrong{Service}}] \leavevmode
A group of Pods can be organized into a logical unit through the concept of a ‘Service’. Using Services, you can declare configurations that affect an entire group of Pods regardless of how many Pods are in that group or their location in the cluster. You can use Services to expose ports, discover services, configure load balancing, and more.

Similarly, a Loom Service comprises of a bunch of micro-services or Pods. For e.g., Loom Asset Manager Services comprises of AssetDB micro-service, Loom IDM micro-service, Loom audit and logging micro-service etc.  TBV with Loom Architects - need review help here.

\end{description}

\begin{figure}[htbp]
\centering
\capstart

\noindent\sphinxincludegraphics[scale=0.75]{{_static/term_loom_pod}.png}
\caption{Figure: Loom Pod}\label{\detokenize{mcdmp_concepts:id21}}\end{figure}


\section{Loom Role Based Access Control (RBAC)}
\label{\detokenize{mcdmp_concepts:loom-role-based-access-control-rbac}}
The term ‘RBAC’ refers to Role Based Access Control. In Loom context it refers to access to various Loom Services and functionality based on the user role that is accessing the feature.  Loom has the concept of different {\hyperref[\detokenize{mcdmp_concepts:content-loom-roles}]{\sphinxcrossref{\DUrole{std,std-ref}{Loom Roles}}}}. Loom User Interface offers different views to users depending upon the role of the user that logs into the Loom ecosystem.  For example, a Loom Platform Admin would be able to access functionality such as addition of Loom customers whereas a Loom Tenant Admin can only access Loom Dashboard views that allow user management within a Tenant’s organization.

RBAC is a key tenet of Loom multi-tenancy feature.

More details on Loom RBAC TBD - Need Architecture teams’ inputs and help on this.


\section{Loom Job}
\label{\detokenize{mcdmp_concepts:loom-job}}\label{\detokenize{mcdmp_concepts:term-loom-job}}
The term ‘Loom Job’ refers to data analysis jobs such as scanning content repositories for specified information as per the enterprise needs using the {\hyperref[\detokenize{mcdmp_concepts:term-loom-policy}]{\sphinxcrossref{\DUrole{std,std-ref}{Loom Policy}}}} framework. Jobs can have sub-jobs and Loom Job Management Service manages tracking and controlling these jobs.  Jobs can be of various types:
\begin{itemize}
\item {} 
Data Classification Job

\item {} 
Data Discovery Job

\item {} 
Data Scan Job

\item {} 
Delete Job

\end{itemize}

A Job can be in one of the following states at a given instant:
* Queued
* Dispatched
* Running
* Canceled
* Completed
* Failed

Loom provides framework and associated services to enable creation, deletion, scheduling, tracking and management of these jobs. In future releases, support for job suspend and resume may be available.


\section{Loom GDPR Terminology}
\label{\detokenize{mcdmp_concepts:loom-gdpr-terminology}}\label{\detokenize{mcdmp_concepts:term-loom-gdpr}}
This section refers to all the terms used in Loom context related to the General Data Protection Regulation (GDPR).


\subsection{Data Inventory}
\label{\detokenize{mcdmp_concepts:term-data-inv}}\label{\detokenize{mcdmp_concepts:data-inventory}}
The term ‘Data Inventory’ in Loom context refers to a record of all the content repositories or data sources belonging to a Loom Customer enterprise. This record contains associated information related to these content repositories including:
\begin{itemize}
\item {} 
Data Type

\item {} 
Data Collection

\item {} 
Data Processing

\item {} 
Data Transfers

\item {} 
Data Storage

\item {} 
Data Protection

\end{itemize}


\subsection{Data Map or Data Mapping}
\label{\detokenize{mcdmp_concepts:data-map-or-data-mapping}}\label{\detokenize{mcdmp_concepts:term-data-map}}
The term ‘Data Map’ or ‘Data Mapping’ in Loom context refers to creative visualization of all the data in an enterprise {\hyperref[\detokenize{mcdmp_concepts:term-data-inv}]{\sphinxcrossref{\DUrole{std,std-ref}{Data Inventory}}}}. Data Mapping gives a bird’s eye view of key metrics associated with enterprise data assets located on premises or in the cloud. It can help enterprise gain visibility into data assets and address several data management tasks such as identifying the risk involved from ‘Personally Identifiable Information’ (PII) that exists within the enterprise landscape.

\begin{figure}[htbp]
\centering
\capstart

\noindent\sphinxincludegraphics[scale=0.75]{{cls_4_risk_anal}.png}
\caption{Figure: Using Data Mapping to identify risk associated with PII}\label{\detokenize{mcdmp_concepts:id22}}\end{figure}

Loom uses {\hyperref[\detokenize{mcdmp_concepts:term-data-inv}]{\sphinxcrossref{\DUrole{std,std-ref}{Data Inventory}}}} and {\hyperref[\detokenize{mcdmp_concepts:term-data-map}]{\sphinxcrossref{\DUrole{std,std-ref}{Data Map or Data Mapping}}}} to address the following customer user cases:
\begin{itemize}
\item {} 
Compliance.

\item {} 
Privacy statements.

\item {} 
Security.

\item {} 
Responding to customer requests related to data access, audit, legal requirements etc..

\item {} 
Responding to data subject requests.

\item {} 
Breach response/preparation.(Support for breach notification is not available in early release of Loom and Loom Applications such as GDPR Readiness))

\item {} 
Cost savings from consolidation and Data optimizations.

\end{itemize}



\renewcommand{\indexname}{Index}
\printindex
\end{document}