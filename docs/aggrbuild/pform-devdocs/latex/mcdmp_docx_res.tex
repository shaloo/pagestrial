%% Generated by Sphinx.
\def\sphinxdocclass{report}
\documentclass[letterpaper,10pt,english]{sphinxmanual}
\ifdefined\pdfpxdimen
   \let\sphinxpxdimen\pdfpxdimen\else\newdimen\sphinxpxdimen
\fi \sphinxpxdimen=.75bp\relax

\usepackage[utf8x]{inputenc}

\usepackage{cmap}
\usepackage[T1]{fontenc}
\usepackage{amsmath,amssymb,amstext}
\usepackage{babel}
\usepackage{times}
\usepackage[Bjarne]{fncychap}
\usepackage[dontkeepoldnames]{sphinx}

\usepackage{geometry}

% Include hyperref last.
\usepackage{hyperref}
% Fix anchor placement for figures with captions.
\usepackage{hypcap}% it must be loaded after hyperref.
% Set up styles of URL: it should be placed after hyperref.
\urlstyle{same}

\addto\captionsenglish{\renewcommand{\figurename}{Fig.}}
\addto\captionsenglish{\renewcommand{\tablename}{Table}}
\addto\captionsenglish{\renewcommand{\literalblockname}{Listing}}

\addto\captionsenglish{\renewcommand{\literalblockcontinuedname}{continued from previous page}}
\addto\captionsenglish{\renewcommand{\literalblockcontinuesname}{continues on next page}}

\addto\extrasenglish{\def\pageautorefname{page}}



\usepackage{enumitem}

\title{DocX Resources}
\date{Apr 07, 2018}
\release{0.7.0}
\author{Veritas Technologies LLC}
\newcommand{\sphinxlogo}{\sphinxincludegraphics{loom.png}\par}
\renewcommand{\releasename}{Release}
\makeindex

\begin{document}

\maketitle
\sphinxtableofcontents
\phantomsection\label{\detokenize{mcdmp_docx_res::doc}}


DocX is the agile shared document development methodology used to develop all documentation related to Multi-Cloud Data Management Platform. You may wish to refer to \sphinxhref{https://confluence.community.veritas.com/display/EDMP/DocX\%3A+All+Resources}{DocX Developer Resources} in addition to the DocX Process Guide and the DocX Style Guide listed below.

\begin{sphinxShadowBox}
\begin{itemize}
\item {} 
\phantomsection\label{\detokenize{mcdmp_docx_res:id11}}{\hyperref[\detokenize{mcdmp_docx_res:mcdmp-shared-development-of-documentation}]{\sphinxcrossref{MCDMP Shared development of Documentation}}}
\begin{itemize}
\item {} 
\phantomsection\label{\detokenize{mcdmp_docx_res:id12}}{\hyperref[\detokenize{mcdmp_docx_res:what-s-in-a-name}]{\sphinxcrossref{What’s in a name}}}

\item {} 
\phantomsection\label{\detokenize{mcdmp_docx_res:id13}}{\hyperref[\detokenize{mcdmp_docx_res:objective}]{\sphinxcrossref{Objective}}}

\item {} 
\phantomsection\label{\detokenize{mcdmp_docx_res:id14}}{\hyperref[\detokenize{mcdmp_docx_res:shared-doc-development-process-highlights}]{\sphinxcrossref{Shared Doc Development Process Highlights}}}

\item {} 
\phantomsection\label{\detokenize{mcdmp_docx_res:id15}}{\hyperref[\detokenize{mcdmp_docx_res:how-it-works}]{\sphinxcrossref{How it works}}}

\item {} 
\phantomsection\label{\detokenize{mcdmp_docx_res:id16}}{\hyperref[\detokenize{mcdmp_docx_res:how-to-contribute}]{\sphinxcrossref{How to contribute}}}

\item {} 
\phantomsection\label{\detokenize{mcdmp_docx_res:id17}}{\hyperref[\detokenize{mcdmp_docx_res:use-cases-applying-model-to-mcdmp-documentation-workflows}]{\sphinxcrossref{Use Cases: Applying model to MCDMP Documentation Workflows}}}

\item {} 
\phantomsection\label{\detokenize{mcdmp_docx_res:id18}}{\hyperref[\detokenize{mcdmp_docx_res:feedback-on-how-to-improve}]{\sphinxcrossref{Feedback on how to improve}}}

\end{itemize}

\item {} 
\phantomsection\label{\detokenize{mcdmp_docx_res:id19}}{\hyperref[\detokenize{mcdmp_docx_res:loom-doc-style-guide}]{\sphinxcrossref{Loom Doc Style Guide}}}
\begin{itemize}
\item {} 
\phantomsection\label{\detokenize{mcdmp_docx_res:id20}}{\hyperref[\detokenize{mcdmp_docx_res:goals-and-audience}]{\sphinxcrossref{Goals and Audience}}}

\item {} 
\phantomsection\label{\detokenize{mcdmp_docx_res:id21}}{\hyperref[\detokenize{mcdmp_docx_res:how-to-use-this-guide}]{\sphinxcrossref{How to use this Guide}}}

\item {} 
\phantomsection\label{\detokenize{mcdmp_docx_res:id22}}{\hyperref[\detokenize{mcdmp_docx_res:restructured-text-style-guide}]{\sphinxcrossref{Restructured Text Style Guide}}}

\item {} 
\phantomsection\label{\detokenize{mcdmp_docx_res:id23}}{\hyperref[\detokenize{mcdmp_docx_res:documentation-do-s}]{\sphinxcrossref{Documentation Do’s}}}

\item {} 
\phantomsection\label{\detokenize{mcdmp_docx_res:id24}}{\hyperref[\detokenize{mcdmp_docx_res:documentation-do-not-s}]{\sphinxcrossref{Documentation Do not’s}}}

\end{itemize}

\item {} 
\phantomsection\label{\detokenize{mcdmp_docx_res:id25}}{\hyperref[\detokenize{mcdmp_docx_res:loom-doc-templates}]{\sphinxcrossref{Loom Doc Templates}}}
\begin{itemize}
\item {} 
\phantomsection\label{\detokenize{mcdmp_docx_res:id26}}{\hyperref[\detokenize{mcdmp_docx_res:template-loom-service-faq}]{\sphinxcrossref{Template: Loom Service FAQ}}}

\item {} 
\phantomsection\label{\detokenize{mcdmp_docx_res:id27}}{\hyperref[\detokenize{mcdmp_docx_res:template-loom-xyz-service-troubleshooting-guide}]{\sphinxcrossref{Template: Loom XYZ Service Troubleshooting Guide}}}

\end{itemize}

\end{itemize}
\end{sphinxShadowBox}


\chapter{MCDMP Shared development of Documentation}
\label{\detokenize{mcdmp_docx_res:docx-resources}}\label{\detokenize{mcdmp_docx_res:content-all-mcdmp-doc-process}}\label{\detokenize{mcdmp_docx_res:docx-res}}\label{\detokenize{mcdmp_docx_res:mcdmp-shared-development-of-documentation}}
\begin{sphinxShadowBox}
\begin{itemize}
\item {} 
\phantomsection\label{\detokenize{mcdmp_docx_res:id28}}{\hyperref[\detokenize{mcdmp_docx_res:what-s-in-a-name}]{\sphinxcrossref{What’s in a name}}}

\item {} 
\phantomsection\label{\detokenize{mcdmp_docx_res:id29}}{\hyperref[\detokenize{mcdmp_docx_res:objective}]{\sphinxcrossref{Objective}}}

\item {} 
\phantomsection\label{\detokenize{mcdmp_docx_res:id30}}{\hyperref[\detokenize{mcdmp_docx_res:shared-doc-development-process-highlights}]{\sphinxcrossref{Shared Doc Development Process Highlights}}}

\item {} 
\phantomsection\label{\detokenize{mcdmp_docx_res:id31}}{\hyperref[\detokenize{mcdmp_docx_res:how-it-works}]{\sphinxcrossref{How it works}}}

\item {} 
\phantomsection\label{\detokenize{mcdmp_docx_res:id32}}{\hyperref[\detokenize{mcdmp_docx_res:how-to-contribute}]{\sphinxcrossref{How to contribute}}}

\item {} 
\phantomsection\label{\detokenize{mcdmp_docx_res:id33}}{\hyperref[\detokenize{mcdmp_docx_res:use-cases-applying-model-to-mcdmp-documentation-workflows}]{\sphinxcrossref{Use Cases: Applying model to MCDMP Documentation Workflows}}}

\item {} 
\phantomsection\label{\detokenize{mcdmp_docx_res:id34}}{\hyperref[\detokenize{mcdmp_docx_res:feedback-on-how-to-improve}]{\sphinxcrossref{Feedback on how to improve}}}

\end{itemize}
\end{sphinxShadowBox}


\section{What’s in a name}
\label{\detokenize{mcdmp_docx_res:what-s-in-a-name}}
The very first thing we’d like to do as part of MCDMP Shared Doc Development Process building, is to rename TechDocs.  The reason being, Rohini thinks that the name should not get confused with InfoDev as this effort is not to replace InfoDev but to augment it and strengthen it.  The TechDocs initiative is to bring in agile documentation practices from contemporary open source world, leaders in IT into Veritas documentation efforts.

You can participate in TechDocs renaming exercise.  Simply click this link \sphinxhref{https://www.surveymonkey.com/r/P85V5FP}{TechDocs Rename Survey}, and vote.  If you don’t like any of the alternatives listed there, suggest your own.

\begin{figure}[htbp]
\centering
\capstart

\noindent\sphinxincludegraphics[scale=0.75]{{techdocs-rename}.png}
\caption{Figure: Some alternative names for TechDocs}\label{\detokenize{mcdmp_docx_res:id3}}\end{figure}


\section{Objective}
\label{\detokenize{mcdmp_docx_res:objective}}
MCDMP Platform is comprised of a set of micro-services that work in tandem to offer rich functionality for information management.  The overall objective is to be the Platform of choice for Enterprise Information Management solutions from Veritas but also third parties. MCDMP allows end users to utilize its rich functionality through REST APIs. To facilitate the onboarding process, MCDMP TechDocs proposes a shared documentation development process similar to the one followed successfully by several popular open source software projects.

This document outlines MCDMP process for shared development of technical documents including the following:
\begin{itemize}
\item {} 
Getting Started Guide

\item {} 
Installation Guide

\item {} 
Concepts and User Guide

\item {} 
SDK API Documentation

\item {} 
MCDMP FAQ

\item {} 
Others (TBD)

\end{itemize}


\section{Shared Doc Development Process Highlights}
\label{\detokenize{mcdmp_docx_res:shared-doc-development-process-highlights}}
Here are some of key highlights of MCDMP Shared Documentation Development Process.  Please note, this is a draft as of now and we hope to refine it further in the days to come, as we build MCDMP itself.

\begin{figure}[htbp]
\centering
\capstart

\noindent\sphinxincludegraphics[scale=0.75]{{shared-doc-dev-model}.png}
\caption{Figure: Shared Document Development Model for MCDMP}\label{\detokenize{mcdmp_docx_res:id4}}\end{figure}

\begin{figure}[htbp]
\centering
\capstart

\noindent\sphinxincludegraphics[scale=0.75]{{shared-doc-vision}.png}
\caption{Figure: Vision: Shared Document Development Model for MCDMP}\label{\detokenize{mcdmp_docx_res:id5}}\end{figure}

\begin{figure}[htbp]
\centering
\capstart

\noindent\sphinxincludegraphics[scale=0.75]{{shared-doc-arch}.png}
\caption{Figure: Architecture: Shared Document Development Model for MCDMP}\label{\detokenize{mcdmp_docx_res:id6}}\end{figure}


\section{How it works}
\label{\detokenize{mcdmp_docx_res:how-it-works}}
The following figure is a placeholder for now.  It shows how the MCDMP Shared documentation development process is intended to function.

\begin{figure}[htbp]
\centering
\capstart

\noindent\sphinxincludegraphics[scale=0.75]{{shared-doc-howitworks}.png}
\caption{Figure: How it works: Shared Document Development Model for MCDMP}\label{\detokenize{mcdmp_docx_res:id7}}\end{figure}


\section{How to contribute}
\label{\detokenize{mcdmp_docx_res:how-to-contribute}}
In this section, we list some high level bullet points on how various MCDMP personnel can contribute to the document content development initiative as we all progress along  building MCDMP itself.

\begin{figure}[htbp]
\centering
\capstart

\noindent\sphinxincludegraphics[scale=0.75]{{shared-doc-howtocontrib}.png}
\caption{Figure: How to Contrbute: Shared Document Development Model for MCDMP}\label{\detokenize{mcdmp_docx_res:id8}}\end{figure}


\section{Use Cases: Applying model to MCDMP Documentation Workflows}
\label{\detokenize{mcdmp_docx_res:use-cases-applying-model-to-mcdmp-documentation-workflows}}
A process is as good as its functioning.  We have drafted the process and believe it works, as is proven by several open source projects and companies such as Google, Twitter and Microsoft that have adopted similar agile documentation methodologies.  In this section, we list potential use cases for this process, to begin with.  The process must address these satisfactorily and needs to evolve as more use cases for documentation arise in future.

\begin{figure}[htbp]
\centering
\capstart

\noindent\sphinxincludegraphics[scale=0.75]{{shared-doc-usecases}.png}
\caption{Figure: Architecture: Shared Document Development Model for MCDMP}\label{\detokenize{mcdmp_docx_res:id9}}\end{figure}


\section{Feedback on how to improve}
\label{\detokenize{mcdmp_docx_res:feedback-on-how-to-improve}}
Have you tried contributing to TechDocs yet? If not, I’d encourage you to do the same and give the process a shot.  If something needs tweaking or can be made easier, better towards better scalability, document legibility or understanding, feel free to send your suggestions to \sphinxhref{mailto://shaloo.shalini@veritas.com}{TechDocs Support}.


\chapter{Loom Doc Style Guide}
\label{\detokenize{mcdmp_docx_res:loom-doc-style-guide}}
How do we make it really easy and simple for Loom users to adopt the platform?

The very basic ingredient is the Platform itself. It needs to be flexible, scalable, intuitive, easy to deploy and quick to gather information insights. Besides that, one of the key enablers towards facilitating onboarding is Loom user documentation. Documentation that satisfies the following criteria:
\begin{itemize}
\item {} 
Concise,

\item {} 
Current \& Continuously updated,

\item {} 
Consistent,

\item {} 
Clear, and

\item {} 
Comprehensive

\end{itemize}

Loom follows a shared, scalable documentation model whereby several contributors work towards creating documentation that helps Loom user onboarding. In order to allow for more consistency across developer documentation, use this style guide.

If you are not aware of the Loom process for shared development of documentation, you may want to refer to {\hyperref[\detokenize{mcdmp_docx_res:content-all-mcdmp-doc-process}]{\sphinxcrossref{\DUrole{std,std-ref}{MCDMP Shared development of Documentation}}}} before you read this style guide.

REST API Platforms such as Loom are designed to be consumed by APIs! It is important that we ensure Loom clients, or consumers, are able to implement these APIs and understand what is happening, with ease and swiftly. As we build Loom, we want to ensure that we not only provide information on these APIs to help developers integrate / debug connections and solutions around Loom, but also return back relevant data whenever a user makes a call, especially when the API call fails.

A big proportion of Loom Technical documents will come from and will be consumed by developers. The Loom Documentation process is geared for the developers.  You can say, it is documentation of the developers, for the developers and by the developers. The ultimate goal is to make it really easy for developers, who don’t know what our Loom product can do, or are not sold on using it yet, to get engaged and use it faster.

This guide is inspired by \sphinxhref{https://developers.google.com/style/}{Google Style Guide} which is adopted by several open source projects out there. Besides that, there are other references such as \sphinxhref{https://www.mulesoft.com/resources/api/guidelines-api-documentation}{API Documentation Guide}, and \sphinxhref{https://blog.readme.io/the-best-rest-api-template/}{Best REST API Template}. These are in use by many projects and many engineers, who collaborate globally to create world class software and documentation, used across verticals and industries.


\section{Goals and Audience}
\label{\detokenize{mcdmp_docx_res:goals-and-audience}}
The main purpose of this style guide is to clearly list and record simple guidelines that Loom team decides to follow as documentation style during shared document development process.

The guide can help developers, authors, technical writers to form a consistent approach for writing API documentation.

The primary goal of this guide is to codify and record decisions that Google’s Developer Relations group makes about style. The guide can help you avoid making decisions about the same issue over and over, can provide editorial assistance on structuring and writing your documentation, and can help you keep your documentation consistent with our other documentation.

This guide is not intended to be an industry standard, nor does it apply to all Veritas documentation.  This is valid only for Loom related platform and associated solutions.


\section{How to use this Guide}
\label{\detokenize{mcdmp_docx_res:how-to-use-this-guide}}
This guide is a reference document.  You can choose to read it linearly or look up specific issue or topic using the search box.

For issues not covered in this guide, see \sphinxhref{https://developers.google.com/style/}{Google Style Guide} or \sphinxhref{https://developers.google.com/style/resources}{Other Style Guides}.


\section{Restructured Text Style Guide}
\label{\detokenize{mcdmp_docx_res:restructured-text-style-guide}}
You may also want to refer to some of the ‘coding’ standards applied to documentation files developed using Sphinx automation engine and Restructured Text formal.  See \sphinxhref{http://documentation-style-guide-sphinx.readthedocs.io/en/latest/style-guide.html}{Sphinx Documentation Style Guide} for details.


\bigskip\hrule\bigskip



\section{Documentation Do’s}
\label{\detokenize{mcdmp_docx_res:documentation-do-s}}

\begin{savenotes}\sphinxattablestart
\centering
\begin{tabular}[t]{|*{2}{\X{1}{2}|}}
\hline
\sphinxstylethead{\sphinxstyletheadfamily 
Purpose
\unskip}\relax &\sphinxstylethead{\sphinxstyletheadfamily 
Guidelines
\unskip}\relax \\
\hline
Clarity
&
Most of the principles that apply to good technical documentation also apply to
accessible technical documentation:
\begin{itemize}
\item {} 
Use correct grammar and punctuation.

\item {} 
Use active voice and present tense.

\item {} 
Write clear, simple sentences.

\item {} 
Be consistent.

\end{itemize}
\\
\hline
Comprehension
&
Include screenshots!
Users find screenshots very helpful, particularly annotated screenshots.
When using screenshots as figures, remember to circle the relevant buttons, add
arrows pointing out mentioned links, or highlight key sections.
\\
\hline
Common norms
&
Spell out numbers nine and under. Use numerals for 10 and up.
\\
\hline
Common norms
&
Use comma before “and” in a list of several items.
For example:
“This theme is elegant, simple, and easy to use.”
\\
\hline\sphinxstartmulticolumn{2}%
\begin{varwidth}[t]{\sphinxcolwidth{2}{2}}
Loom SDK       \textbar{} If you are documenting a workflow that uses Loom Platform SDK APIs, make sure
Consistency     \textbar{} you include working code samples.
\par
\vskip-\baselineskip\vbox{\hbox{\strut}}\end{varwidth}%
\sphinxstopmulticolumn
\\
\hline
Shared doc
development
norm
&
Do not delete, move or rename any document, page or topic which you do not own
or did not create. Discuss and review as there may be cascading impact and
changes required to be done for the same. Collaborate on such wide impact
documentation changes.
\\
\hline
Cross referencing
common norm
&
Be aware that the content of some pages is included in other pages. Make sure
you put your content in the right place, if it is cross referenced multiple
times and also across guides.  Follow similar principles just as in source code
common files and function norms. This applies in particular to the Concepts and
Usage Guide, FAQ, SDK API documentation.
\\
\hline
Loom Document
Consistency
&
If you are inserting a topic in an already existing piece of content, say a page
or in a new section, ensure that you follow a consistent style, layout, grammar
and format with the rest of the page.
\\
\hline
Loom Version
\& Accuracy
&If your update applies to a specific version of a product or framework add or
update a panel at the top of the page:
\begin{itemize}
\item {} 
Include the product version for which your update is relevant.

\item {} 
Use ‘Available’, ‘Changed’ and ‘Deprecated’.

\end{itemize}

Also, you could use the note directive to highlight the section you have updated
and mention the relevant product version to which the change applies.

\{TBD: Example of the above\}
\\
\hline\begin{description}
\item[{Work in Progress}] \leavevmode
Indicator

\end{description}
&\begin{quote}

In a CI/CD shared documentation development model, it is important to let others
know if your document is workable, usable but not yet reviewed, or ready to be
released to the outside world. In other cases, it could be a piece of work in
progress.

When you are writing a lengthy piece of documentation, you may need to publish
it for internal users to start integrating it, even as doc reviews happen.

Please mark all such drafts or unfinished work, with an
information box at the top of the page. See example below:
\end{quote}

\begin{sphinxadmonition}{danger}{Danger:}
\sphinxstylestrong{Work in Progress}
This page is under construction. You are referring to a draft copy.
Use it with caution.
\end{sphinxadmonition}
\\
\hline\sphinxstartmulticolumn{2}%
\begin{varwidth}[t]{\sphinxcolwidth{2}{2}}
\begin{description}
\item[{Loom Document   \textbar{} For API and product version numbers:}] \leavevmode
\begin{DUlineblock}{0em}
\item[] 
\item[]
\begin{DUlineblock}{\DUlineblockindent}
\item[] * Use earlier instead of lower or below, as in Loom Asset Manager 3.0 and
\item[]
\begin{DUlineblock}{\DUlineblockindent}
\item[] earlier.
\end{DUlineblock}
\item[] * Use later instead of above, as in Loom Asset Manager 7.0 and later.
\item[] * Include a tenths value for Loom API version numbers. For example,
\item[]
\begin{DUlineblock}{\DUlineblockindent}
\item[] “This object is available in API version 40.0 and later.”
\end{DUlineblock}
\end{DUlineblock}
\end{DUlineblock}

\end{description}
\par
\vskip-\baselineskip\vbox{\hbox{\strut}}\end{varwidth}%
\sphinxstopmulticolumn
\\
\hline
\end{tabular}
\par
\sphinxattableend\end{savenotes}


\bigskip\hrule\bigskip



\section{Documentation Do not’s}
\label{\detokenize{mcdmp_docx_res:documentation-do-not-s}}

\begin{savenotes}\sphinxattablestart
\centering
\begin{tabular}[t]{|*{2}{\X{1}{2}|}}
\hline
\sphinxstylethead{\sphinxstyletheadfamily 
Purpose
\unskip}\relax &\sphinxstylethead{\sphinxstyletheadfamily 
Guidelines
\unskip}\relax \\
\hline
No Cryptics
&
No Loom Buzzwords please! Before you use any acronym, expand it for first time
use, with acronym in brackets. For subsequent use, acronym is fine.

\begin{DUlineblock}{0em}
\item[] \sphinxstylestrong{Not recommended}: Use JBMgmtSvc to schedule this task if this policy is
\item[] breached.
\item[] 
\item[] \sphinxstylestrong{Recommeded}: Use Job Management Service (JBMgmtSvc) to schedule this task
\item[] if privacy policy is breached. Refer to JBMgmtSvc details here (refer to link)
\item[] on how to set up job management schedule
\item[] 
\end{DUlineblock}
\\
\hline
No Complicated
Language
&
No long sentences. Think of the twitter 140 characters generation. Avoid choppy
sentences.

\begin{sphinxadmonition}{tip}{Tip:}
Think of your audience
\begin{description}
\item[{Compare the first paragraph below to the second in terms of complexity.}] \leavevmode\begin{itemize}
\item {} 
When you write, imagine you speaking these very same words as if you were
saying these to someone in a conference call or Webinar where no real
real time clarification or feedback is possible by those who’d read it
and try to understand what you wrote at a later point in time.

\item {} 
When you write, imagine your audience reading what you write, at a later
point in time. Face to face conversation allow the luxury of real-time
clarifications, additional cues via facial expressions and tone. Written
word is more powerful and could be confusing as it lacks that luxury.

\end{itemize}

\end{description}
\end{sphinxadmonition}
\\
\hline
Factually Correct
&
Avoid using ‘as of now’, ‘at this time’. Evolving platforms such as Loom will
keep changing frequently to meet user needs. Whatever you write, must be
factually correct for the corresponding Loom features available to users.
\\
\hline
Avoid Repetition
&
Use of repetitive phrases such as ‘To do’, or ‘Here is’, or “You can” at the
start of each sentence is detrimental to reader attention.
\\
\hline
Enterprise
Focus
&
Avoid current pop-culture references.
\\
\hline
Appropriate
Language
&
No Jokes at the expense of customers, competitors or anyone else for that
matter.
\\
\hline
Spoiler Alert
&
Avoid trying to document or talk about future features or solutions or products
or enhancements, even in innocuous ways.
\\
\hline
Ambiguity
&
Avoid the modal verbs such as could, may, might, and should in technical
documentation. Modal verbs are ambiguous and leave the reader wondering what to
do. For details, visit \sphinxhref{https://developer.salesforce.com/docs/atlas.en-us.salesforce\_pubs\_style\_guide.meta/salesforce\_pubs\_style\_guide/style\_can.htm}{Modal Style Guide}.
\\
\hline
\end{tabular}
\par
\sphinxattableend\end{savenotes}


\chapter{Loom Doc Templates}
\label{\detokenize{mcdmp_docx_res:loom-doc-templates}}\label{\detokenize{mcdmp_docx_res:mcdmp-doc-template}}

\section{Template: Loom Service FAQ}
\label{\detokenize{mcdmp_docx_res:template-loom-service-faq}}\label{\detokenize{mcdmp_docx_res:t-faq}}

\subsection{Loom Service Name}
\label{\detokenize{mcdmp_docx_res:loom-service-name}}\label{\detokenize{mcdmp_docx_res:t-faq-service-title}}
Update the Loom service name in the title above. For e.g., XYZ Service.


\subsubsection{What is XYZ Service?}
\label{\detokenize{mcdmp_docx_res:what-is-xyz-service}}\label{\detokenize{mcdmp_docx_res:t-faq-service-what}}
Describe the Loom service in terms of what it does. How does a service user benefit from it?


\subsubsection{Service Dependencies}
\label{\detokenize{mcdmp_docx_res:t-faq-service-dep}}\label{\detokenize{mcdmp_docx_res:service-dependencies}}
If there are any other Loom micro-services that you need to deploy in order to use this service, then list those.  Point to those service FAQs if available.


\subsubsection{When to use this service?}
\label{\detokenize{mcdmp_docx_res:t-faq-service-when}}\label{\detokenize{mcdmp_docx_res:when-to-use-this-service}}
In brief, describe the context when this service is used.  For example, job management service is used in the context of “information classification”, whereby some long running tasks and subtasks need to be managed by the job manager service.


\subsubsection{Service Usage REST API Workflow}
\label{\detokenize{mcdmp_docx_res:t-faq-service-how}}\label{\detokenize{mcdmp_docx_res:service-usage-rest-api-workflow}}
Give a high level snapshot of how a developer (user) can utilize this service and what is a typical API workflows?


\subsubsection{Service FAQ Section}
\label{\detokenize{mcdmp_docx_res:t-faq-service-misc}}\label{\detokenize{mcdmp_docx_res:service-faq-section}}
This section lists Q and A relate to this service.

You may want to refer to the \DUrole{xref,std,std-ref}{srv\_idm\_faq} as an example.

The following question and answers are listed as an example. They may not be relevant to your service. As you reuse this template, replace them with actual ones for your service.

\sphinxstylestrong{What API do I need to call in order to list all the components that are part of Loom control plane, data plane?}

Answer TBD

\sphinxstylestrong{Is there an API provided by Cluster Manager Service that returns health status of a cluster?}

Use clustermgr/v1/cluster/\{clustername\}/health.  Note, this returns health of a specific cluster.

\sphinxstylestrong{If a developer (user) queries an object, is it possible to obtain cluster health as part of JSON query?}

Yes. (This may need to be reviewed and reworded properly in the context when such as need arises.)

\sphinxstylestrong{Is there an API that provides list of all services and nodes corresponding to a given Loom component?}

Use clustermgr/v1/health API to obtain information about operational analytics of VMs and overall cluster.

\sphinxstylestrong{Is there an API to obtain various links between Loom components?}

No API’s defined for that.


\section{Template: Loom XYZ Service Troubleshooting Guide}
\label{\detokenize{mcdmp_docx_res:t-trbs-guide}}\label{\detokenize{mcdmp_docx_res:template-loom-xyz-service-troubleshooting-guide}}
\begin{sphinxShadowBox}
\begin{itemize}
\item {} 
\phantomsection\label{\detokenize{mcdmp_docx_res:id35}}{\hyperref[\detokenize{mcdmp_docx_res:service-issue-lorem-ipsum}]{\sphinxcrossref{Service Issue Lorem Ipsum}}}

\item {} 
\phantomsection\label{\detokenize{mcdmp_docx_res:id36}}{\hyperref[\detokenize{mcdmp_docx_res:another-service-issue-lorem-ipsum}]{\sphinxcrossref{Another Service Issue Lorem Ipsum}}}

\end{itemize}
\end{sphinxShadowBox}

\begin{sphinxadmonition}{warning}{Warning:}
This is a sample template for Loom Troubleshooting Guide.  This document is under review and subject to change during Loom Alpha release timeframe.
\end{sphinxadmonition}


\subsection{Service Issue Lorem Ipsum}
\label{\detokenize{mcdmp_docx_res:service-issue-lorem-ipsum}}\begin{description}
\item[{\sphinxstylestrong{Description}}] \leavevmode
Explain the issue that a Loom Service User might face.

Another para describing the issue in detail. You could also include screenshots or a flowchart highlighting the steps that cause this issue.

\item[{\sphinxstylestrong{Workaround}}] \leavevmode
Describe workaround to address the issue listed above.

Another para that may list details of the workaround.

This is last para of the workaround. You can add screenshots here too. For example, refer to the picture below to highlight some section or steps.

\begin{figure}[htbp]
\centering
\capstart

\noindent\sphinxincludegraphics[scale=1.0]{{t_trbs_sample_ref}.png}
\caption{Figure: Troubleshooting Image Sample}\label{\detokenize{mcdmp_docx_res:id10}}\end{figure}

\item[{\sphinxstylestrong{Other Troubleshooting Tips}}] \leavevmode
List other troubleshooting tips related to this issue or point to other similar issues that may be a good reference.

\end{description}


\subsection{Another Service Issue Lorem Ipsum}
\label{\detokenize{mcdmp_docx_res:another-service-issue-lorem-ipsum}}\begin{description}
\item[{\sphinxstylestrong{Description}}] \leavevmode
Explain the issue that a Loom Service User might face.

\item[{\sphinxstylestrong{Workaround}}] \leavevmode
Describe workaround to address the issue listed above.

\item[{\sphinxstylestrong{Other Troubleshooting Tips}}] \leavevmode
List other troubleshooting tips related to this issue or point to other similar issues that may be a good reference.

\end{description}



\renewcommand{\indexname}{Index}
\printindex
\end{document}